\documentclass[a4paper,12pt]{article}
    
    \usepackage[top=1.5cm, bottom=1.5cm,left=1.5cm,right=1.5cm]{geometry}
    
    \usepackage{t1enc}
    \usepackage[utf8]{inputenc}
    \usepackage[magyar]{babel}

    \usepackage{multicol}
\begin{document}
    \noindent\makebox[\textwidth][c]{\Large A számítástudomány alapjai 2021. I. félév}
    \noindent\makebox[\textwidth][c]{\Large 1. gyakorlat}
    \begin{enumerate}
        \item Morty rettenetesen fél a koronavírustól. Ezért úgy döntött, hogy a lehetséges vakcinák közül (Pfizer, Moderna, AstraZeneca, Jansen, Sputnik, Sinopharm) az összessel beoltatja magát kétszer. Hányféleképpen kérheti ekkor az oltásokat, ha a vakcinák sorrendje számít, de két ugyanolyan típusú vakcinát természetesen nem különböztetünk meg? Summer annyira nem félős, ő nem kér mindegyikből, de legalább egyből igen, és ha valamilyen típusúból kér, akkor mindenképp beoltatja magát azzal kétszer. Hány féle sorrendben kérheti Summer a vakcinákat?
        
        \item A Marvel Cinematic University-n véletlenül rosszul írták ki a Gyógy- és strandfürdők nevű $2$ kredites tárgyat a Neptunban, ezért idén a $9689$ hallgató közül csak egy veheti azt fel. A tárgy során a hallgatónak az univerzum $17$ gyógyfürdőjéből és $17$ strandjából kell választania egyet-egyet és meglátogatni azokat. Az év végén a jegyét ($1$ és $5$ között) a hallgató a fürdőkről megírt esszéjére kapja. A kurzus előadója, Doktor Strange nem szereti a meglepetéseket, ezért az Agamotto szemével betekintett a jövőbe, hogy lássa, melyik hallgató, mely fürdőkről fog írni és milyen jegyet fog kapni az esszéjére.
        \begin{enumerate}
            \item Hány lehetséges kimenetelt látott a Doktor?
            \item Ezek közül hány végződött bukással?
        \end{enumerate}
        Közben sikerült kijavítani a tárgy kiírását, így $1$ helyett ismét $30$ hallgató veheti fel a tárgyat.
        \begin{enumerate}
            \setcounter{enumii}{2}
            \item Hány lehetséges kimenetelt látott a Doktor ebben az esetben?
            \item Ezek közül hány esetben nem bukott meg senki?
            \item Hány esetben nem bukott meg mindenki?
        \end{enumerate}

        \item A rivális Mercedes istálló legyőzésének érdekében egy új stratégiával rukkoltak elő a Ferrari fejesei. Az olasz nagydíjon bevetik a $32$ kiállásos taktikát, melynek alapja, hogy minden kiállásnál az előzőtől különböző gumiszettet tesznek fel az autóra. Hányféle gumistratégiája lehet a Ferrari-nak, ha (egy kiskaput átjátszva) mind a $9$ fajta gumikeverékből végtelen sok áll a rendelkezésükre?
        
        \item A $2019$-es félévben a $42$-es tankörbeli $35$ hallgató összesen $100$ pontot ért el a második számtud ZH-ján (egy hallgató akár $100$ pontot is elérhet). Hányféleképpen történhetett ez meg, ha az egyes ZH pontszámok akármekkora nemnegatív egészek lehetnek?
        
        % \item Van-e olyan egyszerű gráf, aminek a fokszámai
        % \begin{enumerate}
        %     \item $1,2,2,3,3,3$;
        %     \item $1,1,2,2,3,4,4$?
        % \end{enumerate}

        % \item \textbf{[PZH-2016]} Tegyük fel, hogy az $F$ fának csak első-, másod- és harmadfokú csúcsai vannak, utóbbiból pontosan tíz darab. Határozzuk meg $F$ leveleinek (azaz elsőfokú csúcsainak) a számát.

        % \item Találjuk meg (izomorfia erejéig) mindazon egyszerű gráfokat, amelyekre
        % \begin{multicols}{3}
        % \begin{enumerate}
        %     \item $n=5$, $e=2$;
        %     \item $n=5$, $e=3$;
        %     \item $n=5$, $e=7$;
        %     \item $n=4$, $e=5$;
        %     \item $n=5$, $e=8$.
        % \end{enumerate}
        % \end{multicols}

        \item Járvány ütött ki a villanykaron! Az $1000$ hallgató mindegyikén végre kellett hajtani egy polimerlánc alapú és egy ellenanyag alapú tesztet (összesen $2000$ tesztet). Egy hallgatót akkor nyilvánítottak fertőzöttnek, ha legalább az egyik teszteredménye (a $2$ közül) pozitív lett. Hányféleképpen alakulhattak az egyes tesztek eredményei, ha csak annyit tudunk, hogy $500$ hallgatót nyilvánítottak fertőzöttnek?

        \hrule

        \item Gombóc Artúr úgy döntött, hogy fogyókúra gyanánt a következő hétre lecsökkenti a napi csoki adagját $58$-ra. Artúr raktárában a csokiknak $3$-féle alakja lehet (kerek, szögletes, lyukas), és $3$-féle íze lehet (ét, fehér, tej). Bizonyítsuk be, hogy van olyan csoki fajta, amiből legalább $46$-ot meg fog enni Artúr a következő héten!
        
        \item Az idei Tour de Ferencváros igazán nagy népszerűségnek örvendhetett, hiszen $200$ profi kerékpáros indult el rajta.
            \begin{itemize}
                \item Legfeljebb hány szakaszból állhatott a verseny, ha tudjuk, hogy semelyik két szakasz végén nem állt ugyanaz a $3$ ember a dobogón? (Mindegy milyen sorrendben.)
                \item Legfeljebb hány szakaszból állhatott a verseny, ha tudjuk, hogy semelyik két szakasznak végén nem lett ugyanaz az első $10$ befutó? (Most számít a sorrend.)
            \end{itemize}
        
        \item A Kőbánya Open tenisztorna döntőjében Novak Djokovic és Daniil Medvedev csaptak össze. A szettek végeredménye $5-7$, $3-6$, $7-5$, $6-4$, $4-6$ lett (Medvedev javára). Hányféle sorrendben történhetett meg a pontszerzés, ha a tenisz szabályainak értelmében egy szett minimum $6$ pontig és legalább $2$ pontos különbségig megy?

        \item
        A rendszámreform előtt a magyar rendszámok alakja BB-SS-SS volt (B=betű,
        S=számjegy). 
        %(Egyes rendőrségi, honvédségi ill.\ diplomata gépkocsikon ma is látható.) 
        Hányféle rendszámot lehetett kiadni? Mennyit nyertünk az új (ú.n.\ svéd
        típusú) BBB-SSS rendszámok bevezetésével? A holland rendszámok hajdan $XX-YY-ZZ$
        alakúak voltak, ahol $\{X,Y,Z\}=\{B,S\}$. Hány rendszámot lehetett ott kiadni?

        \item
        Hány részhalmaza van egy $n$-elemű halmaznak? Hányféle $n$ hosszúságú $0/1$
        sorozat létezik? Mennyi az olyan $0/1$ sorozatok száma, amelyek pontosan $k$
        db $1$-est tartalmaznak?

        \item
        Ha $n$\ focicsapat körmérkőzéses bajnokságot játszik, akkor hány mérkőzésre
        van szükség? Kieséses rendszerben mennyi a szükséges mérkőzések száma?

        \item
        Hány különböző módon lehet kitölteni egy ötöslottószelvényt? Hány
        $5$-, $4$-, $3$- ill.\ $2$-találatos lesz ezek között a sorsolás után?

        \item
        Hány olyan $10$ hosszú dobássorozat van a dobókockával, melyben a dobott
        számok összege $3$-mal osztható?

        \item
        Az $5$-ös Bummjátékban hány Bumm hangzik el $1$-től $1000$-ig? Hány számra
        mondunk Bummo(ka)t? (Az $5$-ös Bummjátékban egymás után mondják a játékosok a
        számokat $1$-től indulva, azzal a megkötéssel, hogy ha a szám tízes
        számrendszerbeli alakjában van $5$-ös, vagy a kimondandó szám $5$-tel
        osztható, akkor nem szabad kimondani az adott számot, hanem helyette
        "Bumm"(ok)at kell mondani, mégpedig minden $5$-ös számjegyért egyet, és
        az $5$ prímfaktor kitevője számszor is Bummolni kell.)

        \item
        Tudományosan igazolt tény, hogy az atlantiszi országok zászlaja $3$
        vízszintes sávból áll, minden sáv a piros, fehér, zöld, kék, sárga, fekete
        színek valamelyikére van színezve, úgy, hogy a szomszédos sávok különböző
        színűek legyenek. Természetesen különböző országok lobogói egymástól
        különbözőek. Legfeljebb hány ország létezhetett Atlantiszban?  Legfeljebb
        hány olyan ország lehet, melynek zászlajában van piros sáv?

        \item
        Nyolc ember szeretne teniszezni három teniszpályán úgy, hogy az egyik
        pályán párost, a két másikon egyénit játszanak. Hányféleképpen tehetik ezt
        meg, ha a pályákat különbözőknek tekintjük, de ugyanazon pálya két
        térfelét nem különböztetjük meg? (Természetesen az embereket is
        különbözőknek tekintjük, és az is számít, hogy a négy páros meccset játszó
        játékos között ki kinek a partnere.)

        \item
        Hányféleképp osztható egy $30$ fős osztály hat, ötfős
        csapatra?

        \item
        A villamosmérnök szak mind az $556$ hallgatója két-két ZH-t írt: egyet
        számítástudományból, egyet pedig analízisből. Számítástudományból senki sem
        ért el $36$ pontnál többet. Bizonyítsuk be, hogy van négy olyan hallgató,
        akik amellett, hogy ugyanannyi pontot kaptak a számítástudomány ZH-jukra,
        analízisből is egyforma osztályzatot szereztek.

        \item \textbf{[ZH-2016]}
        Hány különböző módon lehet
        a METAMATEMATIKA szó betűit egy kör mentén úgy elrendezni, hogy mind a $14$
        betűt pontosan egyszer használjuk fel? Két felírást akkor tekintünk
        azonosnak, ha egyik a másikból egy forgatással megkapható.
        (Nem kell kiszámítani a pontos eredményt: elég egy zárt formula,
        ami mutatja, hogy egy alapműveleteket ismerő számológéppel hogyan
        kapható ez meg.)

        % \item \textbf{[PZH-2016]} Hányféle útvonalon tudunk eljutni a síkon a $(0,0)$ koordinátájú pontból a $(9,10)$ koordinátájú pontba úgy, hogy az útvonal minden pontjának valamelyik koordinátája egész legyen, továbbá az út során sosem távolodhatunk a célponttól?

        % \item \textbf{[ZH-2015]} Hányféleképp lehet $5$ házaspárt leültetni egy $10$ székből álló széksorba, ha a házastársaknak közvetlenül egymás mellé kell ülniük? Mi a válasz $13$ székre?

        % \item \textbf{[PZH-2015]} Az ébredő erő bemutatóját $7$ mikulás nézi meg a krampuszával. Úgy szeretnének leülni egy $14$ székből álló sorba, hogy ne üljön minden mikulás a saját krampusza mellett. Hányféleképp tehetik ezt meg? (A $7$ mikulás és a $7$ krampusz is egymástól jól megkülönböztethető.)

        % \item \textbf{[ZH-2017]} Az osztályba járó $15$ fiú és $15$ lány közül hányféleképp választható olyan $10$ fős küldöttség, amelyikben legalább két lány és legalább két fiú van?

        % \item \textbf{[PZH-2017]} Hányféleképp tölthető ki egy ötöslottószelvény úgy, hogy a lehetséges $90$ számból legalább egy $10$-zel oszthatóra is tippeljünk?

    \end{enumerate}
\end{document}