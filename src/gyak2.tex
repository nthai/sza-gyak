\documentclass[a4paper,12pt]{article}
    
    \usepackage[top=1.5cm, bottom=1.5cm,left=1.5cm,right=1.5cm]{geometry}
    
    \usepackage{t1enc}
    \usepackage[utf8]{inputenc}
    \usepackage[magyar]{babel}
    \usepackage{caption}
    \usepackage{subcaption}
    
    \usepackage{standalone}
    \usepackage{tikz}
    \usetikzlibrary{positioning, graphs}
    \usetikzlibrary{graphs.standard}
    \usetikzlibrary{arrows.meta}

    \usepackage{multicol}
\begin{document}
    \noindent\makebox[\textwidth][c]{\Large A számítástudomány alapjai 2022. I. félév}
    \noindent\makebox[\textwidth][c]{\Large 2. gyakorlat}
    \begin{enumerate}
        \item Keressünk minimális feszítőfát! Hány különböző minimális költségű feszítő fa van az alábbi gráfokban?
        \begin{figure}[!h]
            \centering
            \hfill
            \includestandalone[scale = 0.7]{../grafok/minfesz1} \hfill
            \includestandalone[scale = 0.7]{../grafok/minfesz2} \hfill
            \includestandalone[scale = 0.7]{../grafok/minfeszzh2005} \hfill
        \end{figure}

        \item Válasszunk egy tetszőleges gyökérpontot az alábbi gráfokban és készítsünk feszítőfákat szélességi bejárással. Határozzuk meg az egyes pontok gyökértől való távolságát!
        \begin{figure}[h]
            \centering
            \includestandalone[scale = 0.7]{../grafok/bfs2} \hspace{1in}
            \includestandalone[scale = 0.7]{../grafok/bfs3}
        \end{figure}

        \item Az alábbi feszítőfákat az $a$ csúcsokból indított BFS után kaptuk meg. Hogy nézhetett ki az eredeti gráf az egyes esetekben? 
        \begin{figure}[!h]
            \centering
            \begin{subfigure}{0.2\textwidth}
                \centering
                \includestandalone[width=0.65\textwidth]{../grafok/bfs_after_1}
            \end{subfigure}
            \begin{subfigure}{0.2\textwidth}
                \centering		
                \includestandalone[width=0.65\textwidth]{../grafok/bfs_after_2}
            \end{subfigure}
            \begin{subfigure}{0.2\textwidth}
                \centering
                \includestandalone[width=0.65\textwidth]{../grafok/bfs_after_3}
            \end{subfigure}
            \begin{subfigure}{0.2\textwidth}
                \centering
                \includestandalone[width=0.65\textwidth]{../grafok/bfs_after_4}
            \end{subfigure}
        \end{figure}

        \item \textbf{[ZH-2005]} Hány minimális feszítőfája van a lenti ábrán látható gráf irányítatlan változatának, és mennyi a súlyuk?
        \begin{figure}[!h]
            \centering
            \includestandalone[scale = 0.7]{../grafok/minfeszzh2005}
        \end{figure}

        \item Indítsunk el egy BFS-t (szélességi bejárást) a lenti irányított gráf $s$ csúcsából!
        \begin{figure}[!h]
            \centering
            \includestandalone[scale = 0.7]{../grafok/bfs}
        \end{figure}

        \item Indítsunk BFS-t az alábbi gráfok súlyozatlan változatának az $s$ pontjaiból! Határozzuk meg a legrövidebb utakat ezekben a súlyozatlan gráfokban az $s$ és a $t$ csúcs között!
        \begin{figure}[!h]
            \centering \hfill
            \includestandalone[scale=0.7]{../grafok/dijkstra1}\hfill
            \includestandalone[scale=0.7]{../grafok/dijkstra2}\hfill \hfill
        \end{figure}

        \hrule

        \item \textbf{[PZH-2014]} Az alábbi bal oldali ábrán látható valamely $G$ gráf egy szélességi fája. Honnan indulhatott a bejárás, ha tudjuk, hogy $b$ és $c$ szomszédosak $G$-ben?
        
        \begin{minipage}{6in}
            \centering
            \raisebox{-0.5\height}{\includestandalone{../grafok/bfs_2014pzh}} \hspace{1in}
            \raisebox{-0.5\height}{\includestandalone{../grafok/bfs_2010pzh}}
        \end{minipage}
        
        \item \textbf{[PZH-2015]} A fenti jobb oldali ábrán látható az egyszerű, irányítatlan $G$ gráf $i$ gyökeréből indított szélességi bejárása után kapott $F$ feszítőfa. Tudjuk, hogy az $e$ csúcs $G$-beli fokszáma $7$. Határozzuk meg a $G$ gráf $e$-ből induló éleit.

        \item \textbf{[ZH-2015]} A lenti, bal oldali ábrán látható $G = (V, E)$ gráf élei a felújítandó útszakaszokat jelentik. Minden élén két költség van: az olcsóbbik az egyszerű felújítás költsége, a drágább pedig ugyanez, kerékpárút építéssel. A cél az összes útszakasz felújítása úgy, hogy összefüggő kerékpárúthálózat épüljön ki, amelyen $G$ minden pontja elérhető. Határozzuk meg egy lehető legolcsóbb felújítási tervet, ami teljesíti ezt a feltételt.
        
        \begin{figure}[h]
            \centering
            \begin{subfigure}{0.45\textwidth}
                \centering
                \includestandalone{../grafok/minfesz_2015zh} \hspace{1in}
            \end{subfigure}
            \begin{subfigure}{0.45\textwidth}
                \centering
                \includestandalone{../grafok/minfesz4}
            \end{subfigure}
        \end{figure}
        
        \item A fenti, jobb oldali gráf egy galaxis bolygóinak az úthálózatának egy tervét ábrázolja. Két bolygó akkor van összekötve egy éllel, ha azok közt hiperűr sztrádát tudunk felépíteni, az élek súlya az egyes sztrádák költsége. Mely sztrádákat építsük meg, ha a legkevesebbet szeretnénk költeni, és azt akarjuk, hogy az univerzum bármely bolygójából (közvetlenül vagy közvetve) el lehessen jutni bármely másik bolygóba? Mi a helyzet akkor, ha lehetőségünk van minden bolygón kiépíteni egy csillagkaput, melynek költsége $3$? Ha egy bolygó rendelkezik csillagkapuval, akkor onnan bármely másik szintén csillagkapuval rendelkező bolygóba el tudunk jutni közvetlenül.
        
        

        \item A $V=\{1,2, \ldots, 2n \}$ (számozott) pontokon hány olyan egyszerű $G$ gráf adható meg, melynek $2n-2$ éle van és két egyforma méretű összefüggő komponensből áll?
    \end{enumerate}
\end{document}