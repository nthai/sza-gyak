\documentclass[a4paper,12pt]{article}
    
    \usepackage[top=1.5cm, bottom=1.5cm,left=1.5cm,right=1.5cm]{geometry}
    
    \usepackage{t1enc}
    \usepackage[utf8]{inputenc}
    \usepackage[magyar]{babel}
    \usepackage{caption}
    \usepackage{subcaption}
    
    \usepackage{standalone}
    \usepackage{tikz}
    \usetikzlibrary{positioning, graphs}
    \usetikzlibrary{graphs.standard}
    \usetikzlibrary{arrows.meta}

    \usepackage{multicol}
\begin{document}
    \noindent\makebox[\textwidth][c]{\Large A számítástudomány alapjai 2022. I. félév}
    \noindent\makebox[\textwidth][c]{\Large 4. gyakorlat}
    \begin{enumerate}


        \begin{minipage}{0.65\textwidth}
            \item Adjuk meg az alábbi jobb oldali gráfnak az $a$ pontból indított szélességi fáját, és erre vonatkozóan az előre-, vissza-, és keresztéleket!
            \item Adjuk meg az alábbi jobb oldali gráfnak az $a$ pontból indított mélységi fáját, és erre vonatkozóan az előre-, vissza-, és keresztéleket!
        \end{minipage}
        \begin{minipage}{0.25\textwidth}
            \centering
            \includestandalone[width=\textwidth]{../grafok/dfs1}    
        \end{minipage}

        \item Határozzuk meg az alábbi, bal oldali PERT problémában a legrövidebb végrehajtási időt és a kritikus tevékenységeket. Mik az egyes tevékenységekre a legutolsó időpontok, amikor azokat elkezdve a projekt még épp időben végrehajtható?
        \begin{figure}[!h]
            \centering
            \includestandalone[scale=0.8]{../grafok/pert2} \hspace{2cm}
            \includestandalone[scale=0.8]{../grafok/pert3}
        \end{figure}

        \hrule

        \item Mekkora lehet a fenti jobb oldali PERT diagramban a $p$ legnagyobb értéke, úgy hogy a projekt végrehajtási időtartama minimális legyen?
        
        \item Az alábbi feszítőfákat az $a$ csúcsokból indított DFS után kaptuk meg. Hogy nézhetett ki az eredeti gráf az egyes esetekben? 
        \begin{figure}[!h]
            \centering
            \begin{subfigure}{0.2\textwidth}
                \centering
                \includestandalone[width=0.65\textwidth]{../grafok/bfs_after_1}
            \end{subfigure}
            \begin{subfigure}{0.2\textwidth}
                \centering
                \includestandalone[width=0.65\textwidth]{../grafok/bfs_after_2}
            \end{subfigure}
            \begin{subfigure}{0.2\textwidth}
                \centering
                \includestandalone[width=0.65\textwidth]{../grafok/bfs_after_3}
            \end{subfigure}
            \begin{subfigure}{0.2\textwidth}
                \centering
                \includestandalone[width=0.65\textwidth]{../grafok/bfs_after_4}
            \end{subfigure}
        \end{figure}
        

        \item \textbf{[ZH-2014]} A lenti bal oldali ábrán látható a $G$ gráf egy mélységi fája. Honnan indulhatott a bejárás, ha tudjuk, hogy $b$ és $c$ ill. $a$ és $e$ szomszédosak $G$ben?

        \begin{minipage}{6in}
            \centering
            \raisebox{-0.5\height}{\includestandalone{../grafok/bfs_2014pzh}} \hspace{1in}
            \raisebox{-0.5\height}{\includestandalone{../grafok/bfs_2010pzh}}
        \end{minipage}

        \item \textbf{[ZH-2015]} A fenti jobb oldali ábrán látható az egyszerű, irányítatlan $G$ gráf $i$ gyökeréből indított mélységi bejárás után kapott $F$ feszítőfa. Tudjuk, hogy az $e$ csúcs $G$-beli fokszáma $7$. Határozzuk meg a $G$ gráf $e$-ből induló éleit!
        
        \begin{minipage}{0.6\textwidth}
            \item \textbf{[PZH-2019]} Határozzuk meg az ábrán látható PERT feladathoz tartozó minimális végrehajtási időt. Kritikus-e az $a$ csúcsnak megfelelő tevékenység?
        \end{minipage}
        \begin{minipage}{0.3\textwidth}
            \centering
            \includestandalone[width=\textwidth]{../grafok/kruskal_2019zh}
        \end{minipage}

        \item \label{feladat:zh2011} \textbf{[ZH-2011]} Legyen a $G=(V, E)$ gráf csúcshalmaza $V = \{27,28,\ldots,33\}$, él pedig akkor fusson két csúcs között, ha indexeik relatív prímek: $E = \{ij:(i,j) = 1\}$. Rajzoljuk le $G$ diagramját, indítsunk a $27$ csúcsból mélységi bejárást. Rajzoljuk meg az így kapott fa diagramját, határozzuk meg az egyes csúcsok befejezési számát. (Több lehetséges megoldás esetén csak az egyiket kell megadni.)
        
        \begin{minipage}{0.3\textwidth}
            \centering
            \includestandalone[width=\textwidth]{../grafok/dijkstra_2021zh}
        \end{minipage}
        \hfill
        \begin{minipage}{0.6\textwidth}    
            \item Van-e az ábrán látható $G$ gráfnak olyan feszítőfája, ami az $f$ csúcsból minden más csúcsba tartalmazza a $G$ egy legrövidebb útját? Ha igen, adjunk meg egy ilyen feszítőfát.
            \item Határozzuk meg az ábrán látható $G$ gráfban a $dist(v, a)$ távolságot $G$ minden $v\neq a$ csúcsára. Leheséges-e úgy irányítani a $G$ gráf éleit, hogy minden $v\neq a$ csúcs esetén legyen olyan irányított út $v$-ből $a$-ba, aminek az irányítatlan változata $G$ egy legrövidebb $va$-útja?
        \end{minipage}
        
        \begin{minipage}{0.63\textwidth}
            \item A Wallace részvénytársaság nagy beruházásra készül. Forradalmasítani akarja az új replikáns széria gyártósorait. A projekt tevékenységeit, és a tevékenységek közti időt az alábbi, bal oldali gráf szemlélteti. Az újítással az üzem középső szakaszán bármely két tevékenység közti időt le tudja csökkenteni $2$-vel (az ábrán a duplán húzott élek), viszont minden egyes csökkentés pénzbe kerül. Mely élek mentén és mennyivel érdemes csökkenteni az időt, ha a lehető legtöbbet szeretnénk gyorsítani a gyártáson, ugyanakkor fölöslegesen nem szeretnénk pénzt kidobni.
        \end{minipage}\hfill
        \begin{minipage}{0.3\textwidth}
            \centering
            \includestandalone[width=\textwidth]{../grafok/pert5}
        \end{minipage}
    \end{enumerate}
\end{document}