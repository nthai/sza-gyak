\documentclass[a4paper, 12pt]{article}
    
    \usepackage[top=2.5cm,bottom=1.5cm,left=1cm,right=1cm]{geometry}
    
    \usepackage[utf8]{inputenc}
    \usepackage[magyar]{babel}
    
    \usepackage{standalone}
    \usepackage{tikz}
    \usetikzlibrary{positioning, graphs}
    \usetikzlibrary{graphs.standard}
    \usetikzlibrary{decorations.pathmorphing, patterns,shapes}
    \usepackage{alphalph}
    
    \usepackage{caption}
    \usepackage{subcaption}
    \usepackage{amsmath}
    
    \begin{document}
        \noindent\makebox[\textwidth][c]{\Large A számítástudomány alapjai 2020. I. félév}
        \noindent\makebox[\textwidth][c]{\Large 10. gyakorlat}
        
        % \noindent\centering\parbox[c][5em][c]{0.8\textwidth}{\centering Páros gráfok, párosítások, Hall-tétel, Algoritmus maximális párosítás keresésére páros gráfban. Független/lefogó pont-/élhalmazok, Kőnig és Gallai tételei.}
        
        \begin{enumerate}
            \item Keressünk az alábbi páros gráfokban maximális párosítást az alternáló utas algoritmus segítségével.
            \begin{figure}[!ht]
                \centering
                \begin{subfigure}{0.48\textwidth}
                    \centering
                    \includestandalone[scale=1.2]{../grafok/maxpar1}
                \end{subfigure}
                \begin{subfigure}{0.48\textwidth}
                    \centering
                    \includestandalone[scale=1.2]{../grafok/maxpar4}
                \end{subfigure}
            \end{figure}
            \item Síkbarajzolhatóak-e az alábbi gráfok?
            \begin{figure}[!ht]
                \centering
                \includestandalone{../grafok/sik1}
                \includestandalone{../grafok/sik2}
                \includestandalone{../grafok/sik3}
                \includestandalone{../grafok/sik4}
                \includestandalone{../grafok/sik5}
                \includestandalone{../grafok/sik6}
            \end{figure}
            \item Egy konvex test minden lapja négyszög vagy nyolcszög és minden pontban pontosan $3$ lap találkozik. Mennyi a négyszög- és nyolcszöglapok számának különbsége?
            \item Egy kiránduláson $n$ házaspár vesz részt, és közöttük kellene elosztani $2n$ különböző csokoládét úgy, hogy mindenki egyet kapjon. Tudjuk, hogy minden részvevő legalább $n$ fajtát szeret a $2n$-féle csokoládé közül, és az is teljesül, hogy minden csokoládét szereti minden házaspárnak legalább az egyik tagja. Bizonyítsuk be, hogy ekkor kioszthatók úgy a csokoládék, hogy mindenki olyat kapjon, amit szeret.
            \hrule
            \item Az alábbi (páros) gráfokban valaki a hullámos vonalak mentén hozott létre lehetséges párosításokat. Próbáljunk meg ezeken javítani az alternáló utas algoritmus segítségével, ha lehet. Vizsgáljuk meg, hogy mely gráfban teljesül a Hall-feltétel a fölső csúcsok osztályára, azaz a fölső csúcsok összes $X$ részhalmazára nézzük meg azoknak az $N(X)$ szomszédainak a méretét. Ha nem teljesül, keressünk olyan $X$-et, amire $N(X)$ mérete kisebb $X$ méreténél.
            \begin{figure}[ht]
                \centering
                \begin{subfigure}{0.3\textwidth}
                    \centering
                    \includestandalone[width=0.8\textwidth]{../grafok/maxpar3}
                \end{subfigure}
                \begin{subfigure}{0.3\textwidth}
                    \centering
                    \includestandalone[width=0.8\textwidth]{../grafok/maxpar2}
                \end{subfigure}
                \begin{subfigure}{0.3\textwidth}
                    \centering
                    \includestandalone[width=0.8\textwidth]{../grafok/maxpar5}
                \end{subfigure}
            \end{figure}
            \item Az alábbi gráfok közül melyek topologikusan izomorfak?
			\begin{figure}[!ht]
				\centering
				\begin{subfigure}{0.15\textwidth}
					\centering
					\includestandalone{../grafok/topizo1}
					\caption{}
				\end{subfigure}
				\begin{subfigure}{0.15\textwidth}
					\centering
					\includestandalone{../grafok/topizo2}
					\caption{}
				\end{subfigure}
				\begin{subfigure}{0.15\textwidth}
					\centering
					\includestandalone{../grafok/topizo3}
					\caption{}
				\end{subfigure}
				\begin{subfigure}{0.15\textwidth}
					\centering
					\includestandalone{../grafok/topizo4}
					\caption{}
				\end{subfigure}
				\begin{subfigure}{0.15\textwidth}
					\centering
					\includestandalone{../grafok/topizo5}
					\caption{}
				\end{subfigure}
				\begin{subfigure}{0.15\textwidth}
					\centering
					\includestandalone{../grafok/topizo6}
					\caption{}
				\end{subfigure}
			\end{figure}
                
            \item Ujjgyakorlatok (\textit{Études}):
            \begin{itemize}
                \item Egy összefüggő, egyszerű gráf síkbarajzoltja $20$ csúcsból és $19$ élből áll. Hány tartománya van?
                \item Egy egyszerű gráfnak $8$ csúcsa van, melyek közül $3$ izolált pont. A maradék $5$ csúcs $4$ tartományra osztja a síkot, hány éle lehet?
                \item Egy $8$ csúcsú összefüggő egyszerű gráf $5$ tartományra osztja a síkot. Hány éle van?
                \item Egy $6$ csúcsú összefüggő egyszerű gráfnak $13$ éle van. Hány tartományra osztja a síkot a gráf síkbarajzoltja?
                \item Egy $6$ csúcsú összefüggő egyszerű gráfnak $15$ éle van. Hány tartományra osztja a síkot a gráf síkbarajzoltja?
                \item Egy egyszerű gráfnak $8$ csúcsa van, melyek közül $3$ izolált pont. A maradék $5$ csúcsnak $10$ éle van. Hány tartományra oszthatja a síkot a gráfnak egy lerajzolása?
            \end{itemize}
        
            \item Töröljünk ki az alábbi gráfokból csúcsokat vagy éleket úgy, hogy a megmaradt gráf topologikusan izomorf legyen egy $K_5$-tel vagy egy $K_{3,3}$-mal, ahol a $K_5$ vagy $K_{3,3}$ csúcsait a fekete négyzetek jelölik!
            \begin{figure}[ht]
                \centering
                \begin{subfigure}{0.2\textwidth}
                    \centering
                    \includestandalone{../grafok/sik_basic3}
                \end{subfigure}
                \begin{subfigure}{0.2\textwidth}
                    \centering
                    \includestandalone{../grafok/sik_basic1}
                \end{subfigure}
                \begin{subfigure}{0.2\textwidth}
                    \centering
                    \includestandalone{../grafok/sik_basic2}
                \end{subfigure}
                \begin{subfigure}{0.2\textwidth}
                    \centering
                    \includestandalone{../grafok/sik_basic4}
                \end{subfigure}
            \end{figure}
            \hrule
            \item Tfh $G$ egyszerű, $|V(G)|=2000$ és $\tau(G)=678$. Igazoljuk, hogy $G$-ben nincs teljes párosítás! 
            \item \textbf{[PZH-2014]} Tegyük fel, hogy a $88$ pontú $G$ páros gráfban $\alpha(G) = 44$. Igazoljuk, hogy $G$-re teljesül a Hall feltétel, azaz $|X| \leq |N(X)|$ az $A$ színosztály minden $X$ részhalmaza esetén.
            \item \textbf{[ZH-2014]} Tegyük fel, hogy a $G$ egyszerű páros gráf $A$ színosztálya $28$, a $B$ színosztálya $33$ pontú. Tegyük fel, hogy a $B$ színosztálynak valamely $Y$ részhalmazára $|Y|=18$ és $|N(Y)|=12$. Mutassuk meg, hogy az $A$ színosztályra nem teljesül a Hall feltétel, azaz létezik olyan $X \subseteq A$ halmaz, melyre $|N(X)| < |X|$.
            \item \textbf{[ZH-2015]} Tegyük fel, hogy a $G$ egyszerű, páros gráf mindkét színosztálya egyenként $99$ pontot tartalmaz, az $A$ színosztályban minden pont foka legalább $66$, $B$-ben pedig legalább $33$. Mutassuk meg, hogy $G$-nek van teljes párosítása.
            \item \textbf{[PZH-2015]} Tegyük fel, hogy $G=(A,B;E)$ egyszerű, páros gráf $A$ színosztályában $99$ csúcs van, ezek bármelyikének a fokszáma legalább $33$, de $A$-ban van $66$ olyan csúcs, amelyek bármelyikének foka legalább $66$. Sőt, $a$ tartalmaz $33$ olyan csúcsot is, amelyek mindegyikéből legalább $99$ él indul. Mutassuk meg, hogy $G$-nek van $A$-t fedő párosítása.
            \item \textbf{[ZH]} Síkbarajzolhatóak-e az alábbi gráfok?
            \begin{figure}[!ht]
                \centering
                \begin{subfigure}{0.2\textwidth}
                    \centering				
                    \includestandalone[height = 3cm]{../grafok/kromatikus2}
                    \caption{ZH - 2010}
                \end{subfigure}
                \begin{subfigure}{0.2\textwidth}
                    \centering					
                    \includestandalone[height = 3cm]{../grafok/sik_2015zh}
                    \caption{ZH - 2015}
                \end{subfigure}
                \begin{subfigure}{0.25\textwidth}
                    \centering					
                    \includestandalone[height = 3cm]{../grafok/sik_2011pzh}
                    \caption{PZH - 2011}
                \end{subfigure}
                \begin{subfigure}{0.2\textwidth}
                    \centering					
                    \includestandalone[height = 3cm]{../grafok/sik_2012zh}
                    \caption{ZH - 2012}
                \end{subfigure}
            \end{figure}
            \item Igazoljuk, hogy ha a egy egyszerű $G$ gráfnak legalább $11$ csúcsa van, akkor $G$ és $\bar{G}$ közül legalább az egyik nem síkbarajzolható.
            \item Hány csúcsa van egy olyan összefüggő síkbarajzolható gráfnak, aminek három háromszöglapja, három négyszöglapja és egy ötszöglapja van?
        \end{enumerate}
    
    \end{document}