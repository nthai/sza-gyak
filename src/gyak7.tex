\documentclass[a4paper,12pt]{article}
    
    \usepackage[top=1.5cm, bottom=1.5cm,left=1.5cm,right=1.5cm]{geometry}
    \usepackage{graphicx}
    
    \usepackage{t1enc}
    \usepackage[utf8]{inputenc}
    \usepackage[magyar]{babel}
    \usepackage{caption}
    \usepackage{subcaption}
    
    \usepackage{standalone}
    \usepackage{tikz}
    \usetikzlibrary{positioning, graphs}
    \usetikzlibrary{graphs.standard}
    \usetikzlibrary{patterns}
    \usetikzlibrary{arrows.meta}
    
    \usepackage{amssymb}
    \usepackage{amsmath}

    \begin{document}
        \noindent\makebox[\textwidth][c]{\Large A számítástudomány alapjai 2020. I. félév}
        \noindent\makebox[\textwidth][c]{\Large 7. gyakorlat}
        \begin{enumerate}

            \item Állapítsuk meg, hogy van-e az alábbi gráfokban Hamilton-kör! Ha van, mutassuk meg, ha nincs, bizonyítsuk be.
            \begin{figure}[!h]
                \centering
                \begin{subfigure}{0.3\textwidth}
                    \centering
                    \includestandalone[scale=0.55]{../grafok/hamilton1}
                \end{subfigure}
                \begin{subfigure}{0.3\textwidth}
                    \centering
                    \includestandalone[scale=0.55]{../grafok/hamilton2}
                \end{subfigure}
            \end{figure}

            \item \textbf{[ZH-2001]} Legyen $G$ a $\{p_1, p_2, \ldots, p_{2001}\}$ ponthalmazon az az egyszerű gráf, amire ($p_i p_j \in E(G) \Leftrightarrow |i-j| \le 2$). Van-e a $G$-ben Euler-körséta, Euler-séta, Hamilton-kör ill. Hamilton-út?

            \item \textbf{[ZH-2009]} Legyenek $G_1(V,E_1)$ és $G_2(V,E_2)$ olyan egyszerű, összefüggő gráfok a $V$ ponthalmazon, amelyekben van Euler-kör. Konstruáljuk meg a $G_3(V,E_3)$ gráfot úgy, hogy $E_3=(E_1\setminus E_2 \cup E_2 \setminus E_1)$, vagyis $G_3$-ban akkor szomszédos két pont, ha $G_1$-ben, vagy $G_2$-ben szomszédosak, de mindkettőben nem. Igaz-e, hogy ha $G_3$ összefüggő, akkor van benne Euler-kör?

            \item \textbf{[ZH-2015]} $222$ politikus mindegyike legalább $133$ másikat ismer, akik közül legfeljebb $22$-t utál. Az ismeretség és az utálat is kölcsönös. Bizonyítsuk be, hogy a $222$ politikus úgy tudja élő lánccal körülvenni a Tüskecsarnokot, hogy a szomszédos láncszemek ismerjék, de ne utálják egymást.

            \item \textbf{[PZH-2010]} Egy $12$ egység hosszú drótból szeretnénk elkészíteni egy egységkocka élvázát, úgy, hogy a kocka csúcsainál forrasztunk. Legkevesebb hány darabra kell felvágni ehhez az eredeti drótunkat? Mi a válasz akkor, ha a testátlóknak is benne kell lenniük az élvázban, és persze a kiindulási drótunk is $4$ testátlónyival hosszabb?

            \item \textbf{[ZH-2011]} Tegyük fel, hogy a $16$ pontú $K_{16}$ teljes gráf éleit $4$-féle színnel színeztük ki úgy, hogy minden egyes színre az adott színnel színezett élek reguláris gráfot alkotnak $K_{16}$ csúcsain. Igazoljuk, hogy a kiválasztott két szín a $4$ közül úgy, hogy az e két színnel színezett élekből található $K_{16}$-nak Hamilton-köre.

            \item \textbf{[PZH-2011]} Tudjuk, hogy a $99$ csúcsú, egyszerű $G$ gráf maximális fokszáma $\Delta(G)=30$, másrészt $G$-nek van Euler-köre. Mutassuk meg, hogy a $\bar{G}$ komplementergráfnak is van Euler-köre.

            \item \textbf{[PPZH-2011]} Legyen $G$ olyan véges gráf, aminek $C$ egy Hamilton köre. Tegyük fel, hogy a $G-C$ gráfnak van Euler-köre. Mutassuk meg, hogy ekkor a $G$ gráfnak is van Euler-köre.

            \item \textbf{[ZH-2012]} Tegyük fel, hogy az egyszerű $G$ gráfnak $100$ csúcsa van, ezek közül $u$ és $v$ foka $45$, a többi csúcsé pedig legalább $55$. Igazoljuk, hogy $G$-ben van Hamilton út.

            \item \textbf{[ZH-2008]} Tegyük fel, hogy az $n$ csúcsú, irányítatlan $G$ gráf bármelyik csúcsából $G$-nek legfeljebb $\frac{n-2}{2}$ másik csúcsba lehet úton eljutni. Igazoljuk, hogy a $\bar{G}$ komplementergráfnak van Hamilton köre.

            \item \textbf{[PZH-2017]} A $G$ egyszerű gráfnak $33$ piros, $777$ fehér, $333$ zöld, $77$ sárga csúcsa van. Két csúcs között pontosan akkor fut él, ha azok különböző színűek. Behúzható-e $G$-be néhány további él, úgy, hogy olyan egyszerű gráfot kapjunk, aminek van Euler-sétája?

            \item \textbf{[ZH-2016]} Kritikus a helyzet: Abszurdisztán fővárosát, Mutyipusztát savköpő menyétek inváziója fenyegeti. A jobb oldali ábrán látható a főváros térképe: az egyes utak mellett álló számok az adott útvonal hosszát jelölik. A veszélyt -- mint mindig -- most is az ügyeletes szuperhős, Órarugógerincű Felpattanó hárítja el. Mesteri tervének végrehajtása mellett (miszerint helikopterről lúgot permetezve semlegesíti a betolakodókat) még ebben a válságos pillanatban is a közvagyon megóvása a legfőbb célja. Ezért amellett, hogy minden utcát végigpermetez és visszatér a szabadon választott kiindulási pontra, szeretné egyúttal minimalizálni a lerepült össztávot is. Segítsünk Órarugógerincűnek abban, hogyan válasszon útvonalat!
            \begin{figure}[h]
                \centering
                \includestandalone{../grafok/euler_2016zh}
            \end{figure}

            \item \textbf{[Codeforces \#288]} Amíg apukája dolgozott, Tanya úgy döntött, hogy játszik egyet az apja jelszavával. A jelszó egy $n+2$ karakterből álló sorozat. Tanya felírta a karaktersorozat mind az $n$ darab $3$ karakterből álló részsorozatát külön papírcetlikre, majd az eredeti jelszót kidobta. A kislány később rájött, hogy ezt nem kellett volna megtennie, ezért megpróbálja visszaállítani az eredeti jelszót a $3$ betűs részletekből. Hozzunk létre olyan eljárást, ami segít Tanyának előálltani egy jelszót a megfelelő karakterhármasoknak, vagy megállapítani azt, ha ez nem lehetséges. Állítsunk elő egy lehetséges jelszót a következő karakterhármasokból: $\{ 010, 101, 110, 001, 101, 010, 010, 101, 011, 010, 111,101, 000 \}$.
        \end{enumerate}
    \end{document}