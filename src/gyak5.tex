\documentclass[a4paper,12pt]{article}
    
\usepackage[top=1.5cm, bottom=1.5cm,left=1.5cm,right=1.5cm]{geometry}
\usepackage{graphicx}

\usepackage{t1enc}
\usepackage[utf8]{inputenc}
\usepackage[magyar]{babel}
\usepackage{caption}
\usepackage{subcaption}

\usepackage{standalone}
\usepackage{tikz}
\usetikzlibrary{positioning, graphs}
\usetikzlibrary{graphs.standard}
\usetikzlibrary{patterns}
\usetikzlibrary{arrows.meta}

\usepackage{amssymb}
\usepackage{amsmath}

\begin{document}
    \noindent\makebox[\textwidth][c]{\Large A számítástudomány alapjai 2020. I. félév}
    \noindent\makebox[\textwidth][c]{\Large 5. gyakorlat}
    \begin{enumerate}
        \item Indítsunk BFS-t az alábbi gráfok súlyozatlan változatának az $s$ pontjaiból! Határozzuk meg a legrövidebb utakat az alábbi súlyozott gráfokban az $s$ és a $t$ csúcs között!
        \begin{figure}[!h]
            \centering \hfill
            \includestandalone[scale=0.7]{../grafok/dijkstra1}\hfill
            \includestandalone[scale=0.7]{../grafok/dijkstra2}\hfill \hfill
        \end{figure}
        \item Az alábbi feszítőfákat az $a$ csúcsokból indított BFS után kaptuk meg. Hogy nézhetett ki az eredeti gráf az egyes esetekben? 
        \begin{figure}[!h]
            \centering
            \begin{subfigure}{0.2\textwidth}
                \centering
                \includestandalone[width=0.65\textwidth]{../grafok/bfs_after_1}
            \end{subfigure}
            \begin{subfigure}{0.2\textwidth}
                \centering		
                \includestandalone[width=0.65\textwidth]{../grafok/bfs_after_2}
            \end{subfigure}
            \begin{subfigure}{0.2\textwidth}
                \centering
                \includestandalone[width=0.65\textwidth]{../grafok/bfs_after_3}
            \end{subfigure}
            \begin{subfigure}{0.2\textwidth}
                \centering
                \includestandalone[width=0.65\textwidth]{../grafok/bfs_after_4}
            \end{subfigure}
        \end{figure}

        % \item Határozzuk meg az alábbi bal oldali gráfban a legrövidebb utakat az $s$ és a többi csúcs között a Bellman-Ford algoritmussal!
        % \begin{figure}[h]
        %     \centering
        %     \begin{subfigure}{0.4\textwidth}
        %         \centering
        %         \includestandalone{../grafok/bellmanford1}
        %     \end{subfigure}
        %     \begin{subfigure}{0.4\textwidth}
        %         \centering
        %         \includestandalone{../grafok/floyd1}
        %     \end{subfigure}
        % \end{figure}
        
        % \item Határozzuk meg a fenti jobb oldali gráfban az egyes csúcsok közti legrövidebb utakat a Floyd algoritmus segítségével!
        
        
        \item Az alábbi ábrákon gráfok részletei láthatóak, amin épp a Dijkstra algoritmust hajtjuk végre. Az $a$, $b$ és $c$ csúcsokat már bevettük a megvizsgált csúcsok halmazába, a csúcsok mellett zárójelben vannak feltüntetve, hogy a kezdőponttól milyen távol vannak. Melyik csúcsot fogja bevenni következőnek a Dijkstra algoritmus és milyen távolsággal?
        \begin{figure}[!h]
            \centering
            \begin{subfigure}{0.24\textwidth}
                \centering
                \includestandalone[width=0.9\textwidth]{../grafok/dijkstra_small_7}
            \end{subfigure}
            \begin{subfigure}{0.24\textwidth}
                \centering
                \includestandalone[width=0.9\textwidth]{../grafok/dijkstra_small_8}
            \end{subfigure}
            \begin{subfigure}{0.24\textwidth}
                \centering
                \includestandalone[width=0.9\textwidth]{../grafok/dijkstra_small_9}
            \end{subfigure}
            \begin{subfigure}{0.24\textwidth}
                \centering
                \includestandalone[width=0.9\textwidth]{../grafok/dijkstra_small_10}
            \end{subfigure}
        \end{figure}

        \item Az alábbi $K_4$ (irányítatlan) gráfok éleire írjunk pozitív egész élsúlyokat úgy, hogy ha a Dijkstra algoritmust $s$-ből indítjuk, akkor a vastagon szedett élek mentén adja a legrövidebb utakat; a csúcsokat minden esetben (egyértelműen) $1$, $2$, $3$ sorrendben látogatja meg; az élsúlyok összege a lehető legkisebb.
        \begin{figure}[!h]
            \centering
            \begin{subfigure}{0.15\textwidth}
                \centering
                \includestandalone{../grafok/dijkstra_small_1}
            \end{subfigure}
            \begin{subfigure}{0.15\textwidth}
                \centering
                \includestandalone{../grafok/dijkstra_small_2}
            \end{subfigure}
            \begin{subfigure}{0.15\textwidth}
                \centering
                \includestandalone{../grafok/dijkstra_small_3}
            \end{subfigure}
            \begin{subfigure}{0.15\textwidth}
                \centering
                \includestandalone{../grafok/dijkstra_small_4}
            \end{subfigure}
            \begin{subfigure}{0.15\textwidth}
                \centering
                \includestandalone{../grafok/dijkstra_small_5}
            \end{subfigure}
            \begin{subfigure}{0.15\textwidth}
                \centering
                \includestandalone{../grafok/dijkstra_small_6}
            \end{subfigure}
        \end{figure}

        \hrule

        \item \textbf{[PZH-2014]} Az alábbi bal oldali ábrán látható valamely $G$ gráf egy szélességi fája. Honnan indulhatott a bejárás, ha tudjuk, hogy $b$ és $c$ szomszédosak $G$-ben?
        
        \begin{minipage}{6in}
            \centering
            \raisebox{-0.5\height}{\includestandalone{../grafok/bfs_2014pzh}} \hspace{1in}
            \raisebox{-0.5\height}{\includestandalone{../grafok/bfs_2010pzh}}
        \end{minipage}
        
        \item \textbf{[PZH-2015]} A fenti jobb oldali ábrán látható az egyszerű, irányítatlan $G$ gráf $i$ gyökeréből indított szélességi bejárása után kapott $F$ feszítőfa. Tudjuk, hogy az $e$ csúcs $G$-beli fokszáma $7$. Határozzuk meg a $G$ gráf $e$-ből induló éleit.


        \item \textbf{[PZH-2010]} Adott egy $G$ gráf, az $e$ él hosszát jelölje $l(e)$. Minden él hosszát növeljük meg $2$-vel, azaz legyen $l'(e)=l(e)+2$ minden élre. Tegyük fel, hogy $u$ és $v$ között $P$ egy legrövidebb út az $l'$ élhosszokkal. Igaz-e, hogy $P$ biztosan egy legrövidebb út $u$ és $v$ között az $l$ élhosszokra nézve is?
        
        \item \textbf{[ZH-2014 alapján]} Legyenek a $7$ csúcsú $G$ gráf pontjai $v_1$, $v_2$, $v_3$, $v_4$, $v_6$, $v_8$ és $v_9$, valamint akkor legyen $v_i$ és $v_j$ szomszédos, ha $i$ és $j$ relatív prímek. Ekkor a $v_iv_j$ él hosszúsága $|i-j|$. Határozzuk meg a $v_1$ csúcsból minden más csúcsba egy-egy legrövidebb utat.
        
        \item \textbf{[PZH-2014]} Legyen $V(G) = \{v_3, v_4, \ldots, v_{10}\}$, és $v_iv_j \in E(G)$, ha $i$ és $j$ nem relatív prímek, azaz van $1$-nél nagyobb közös osztójuk. Legyen a $v_iv_j$ él hossza $\min(i,j) - 1$. Határozzuk meg a $v_5$ csúcsból minden más csúcsba egy-egy legrövidebb utat, ha van. 
        
        \item \textbf{[ZH-2008]} Határozzuk meg a lenti bal oldali gráfban az élsúlyokat úgy, hogy a Dijkstra algoritmus rossz eredményt adjon!
        \begin{figure}[h]
            \centering
            \begin{subfigure}{0.4\textwidth}
                \centering
                \includestandalone{../grafok/dijkstra3}
            \end{subfigure}
        \end{figure}

        \item \label{feladat:zh2011} \textbf{[ZH-2011]} Legyen a $G=(V, E)$ gráf csúcshalmaza $V = \{27,28,\ldots,33\}$, él pedig akkor fusson két csúcs között, ha indexeik relatív prímek: $E = \{ij:(i,j) = 1\}$. Rajzoljuk le $G$ diagramját, indítsunk a $27$ csúcsból szélességi bejárást, valamint határozzuk meg a bejáráshoz tartozó fát és a többi csúcsnak a $27$ csúcstól való távolságát.

        \item KITÖRT AZ AFRIKAI SERTÉSPESTIS! Az alábbi ábrán a város csomópontjai és azok föld alatti összeköttetései találhatóak. A többszáz malac az $a$ pontbeli karanténból kiszabadulva minden lehetséges irányba elkezdett rohanni. A malacoknak egy napba telik, hogy egy összeköttetésen keresztül át tudjanak menni az egyik csomópontból a másikba. Új csomópontba érkezve a malacok megfertőzik az ottani sertésállomány egyedeit, amelyek szintén megvadulnak és kiszabadulva csatlakoznak az ámokfutáshoz.
        
        \begin{minipage}{0.6\textwidth}
            \begin{enumerate}
                \item Hány nap múlva fertőződik meg a város összes sertése?
                \item Mely csomópont(ok) fertőződik(fertőződnek) meg utoljára?
                \item A második nap végén megérkezik Rick Sanchez (C-$137$) és pillanatok alatt összeállítja a sertéspestis ellenszerét (amely többek között tartalmaz kaktusz, golden retriever, cápa és dinoszaurusz DNS-t is). Az ellenszer egy vírusfelhőként szin-tén a föld alatti utakon terjed, de kétszer olyan gyorsan. A hatóanyag egyből gyógyít és immunissá tesz, viszont hátulütője, hogy cronenbergekké változtatja a fertőzött malacokat. Lesz-e olyan csomópont, ahol nem változnak cronenberggé a malacok?
            \end{enumerate}
        \end{minipage}
        \begin{minipage}{0.3\textwidth}
            \centering
            \includestandalone[width=\textwidth]{../grafok/undirected_weightless_1}
        \end{minipage}

        % \item Úgy tűnik a Galaktikus Föderáció kezd kilábalni a gazdasági csődjéből (miután sikerült visszaállítani a centralizált galaktikus pénznem, a blemflarck értékét $0$-ról $1$-re). Rick Sanchez azonban ezt nem hagyhatja annyiban, az intergalaktikus terrorista ismét monetáris csapást készül mérni. A föderációs adatbázisokat meghekkelve Rick átállította a galaxis pénzeinek árfolyamát az alábbi táblázat alapján, mely azt írja le, hogy egy adott pénz egységéért mennyit kap egy másikból (pl. itt $16$ flurboért $1$ brapple-t lehet kapni). Rick terve az, hogy ügyes átváltásokkal végtelen sok pénzt fog tudni termelni magának. Sikerülni fog-e ez neki emellett a módosított árfolyam mellett.

        % \begin{table}[h]
        %     \centering
        %     \begin{tabular}{|c|c|c|c|c|c|}
        %         \hline
        %             & Blemflarck & Brapple & Flurbo & Schmeckle & Smidgen \\ \hline
        %         Blemflarck & $1:1$ & $8:1$ & $1:2$ & $4:1$ & $128:1$ \\ \hline
        %         Brapple & $1:4$ & $1:1$ & $16:1$ & $1:2$ & $8:1$ \\ \hline
        %         Flurbo & $4:1$ & $16:1$ & $1:1$ & $4:1$ & $64:1$ \\ \hline
        %         Schmeckle & $4:1$ & $8:1$ & $1:4$ & $1:1$ & $8:1$ \\ \hline
        %         Smidgen & $1:4$ & $1:8$ & $4:1$ & $4:1$ & $1:1$ \\ \hline

        %     \end{tabular}
        % \end{table}
    \end{enumerate}
\end{document}