\documentclass[a4paper, 12pt]{article}
    \usepackage[top=1.5cm, bottom=1.5cm, left=1.5cm, right=1.5cm]{geometry}
    \usepackage{amsmath}
    \usepackage{amssymb}
    \usepackage{subcaption}

    \begin{document}
        \noindent\makebox[\textwidth][c]{\Large A számítástudomány alapjai 2022. I. félév}
        \noindent\makebox[\textwidth][c]{\Large 2. ZH gyakorló feladatsor}

        \begin{enumerate}
            \item A $\sigma$ permutáció a $[3, 6, 4, 1, 8, 7, 5, 0, 2, 9]$ sorozatból a $[2, 0, 8, 1, 6, 5, 3, 4, 7, 9]$ sorozatot állítja elő. Határozzuk meg a $\sigma$ inverziószámát.
            
            \noindent\begin{minipage}{0.45\textwidth}
                \item \textbf{[BSZ1-ZH-2014]} Számítsuk ki az alábbi determináns értékét:
                    \[\left|\begin{array}{rrrrr}
                        1  & -1 & 0 &  3 &  0 \\
                        -1 &  1 & 3 &  6 &  5 \\
                        -3 &  3 & 2 & -3 &  2 \\
                        3  & -1 & 4 &  9 & -8 \\
                        -4 &  4 & 1 & -8 &  3
                    \end{array}\right|\]
            \end{minipage}\hfill
            \noindent\begin{minipage}{0.45\textwidth}
                \item \textbf{[BSZ1-PZH-2014]} Számítsuk ki az alábbi determináns értékét csak a determináns de-finíciójának felhasználásával:
                \[\left|\begin{array}{rrrrr}
                        0  & 2 &  0 & -3 & 0 \\
                        2  & 9 &  0 &  7 & 3 \\
                        -4 & 5 & -2 & -6 & 8 \\
                        6  & 7 &  0 &  1 & 0 \\
                        -1 & 0 &  0 &  1 & 0
                \end{array}\right|\]
            \end{minipage}
            \item \textbf{[BSZ1-PPZH-2014]} Az alábbi $A$ mátrixra $\det A = 45653$. Mennyi $\det B$ értéke?
            \begin{figure}[h!]
                \centering
                \begin{subfigure}{0.4\textwidth}
                    \centering
                    \[
                        A = \left(\begin{array}{cccccc}
                            9 & 2 & 8 & 1 & 2 & 1 \\
                            4 & 7 & 2 & 7 & 6 & 9 \\
                            6 & 4 & 3 & 1 & 8 & 3 \\
                            8 & 3 & 0 & 4 & 6 & 5 \\
                            2 & 1 & 7 & 3 & 9 & 0 \\
                            0 & 5 & 6 & 9 & 8 & 4 
                        \end{array}\right)    
                    \]
                \end{subfigure}
                \begin{subfigure}{0.4\textwidth}
                    \centering
                    \[
                        B = \left(\begin{array}{cccccc}
                            9 & 4 & 8 & 2 & 2 & 2 \\
                            2 & 7 & 1 & 7 & 3 & 9 \\
                            6 & 8 & 3 & 2 & 8 & 6 \\
                            4 & 3 & 0 & 4 & 3 & 5 \\
                            2 & 2 & 7 & 6 & 9 & 0 \\
                            0 & 5 & 3 & 9 & 4 & 4 
                        \end{array}\right)    
                    \]
                \end{subfigure}
            \end{figure}
            \item \textbf{[BSZ1-ZH-2014]} Döntsük el, hogy a $p$ valós paraméter milyen értékeire van megoldása az alábbi egyenletrendszernek. Ha van megoldás, adjuk is meg az öszeset.
            \begin{align*}
                x_1 - x_2 + 4x_4 &= -2 \\
                2x_1 - 2x_2 + x_3 + 8x_4 &= -3 \\
                x_1 + x_2 + 6x_3 + 8x_4 &= 2 \\
                3x_1 - 3x_2 + p \cdot x_3 + (p^2 + p + 12)\cdot x_4 &= -6
            \end{align*}
        \end{enumerate}
    \end{document}