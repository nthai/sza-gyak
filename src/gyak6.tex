\documentclass[a4paper,12pt]{article}
    
\usepackage[top=1.5cm, bottom=1.5cm,left=1.5cm,right=1.5cm]{geometry}
\usepackage{graphicx}

\usepackage{t1enc}
\usepackage[utf8]{inputenc}
\usepackage[magyar]{babel}
\usepackage{caption}
\usepackage{subcaption}

\usepackage{standalone}
\usepackage{tikz}
\usetikzlibrary{positioning, graphs}
\usetikzlibrary{graphs.standard}
\usetikzlibrary{patterns}
\usetikzlibrary{arrows.meta}
\usetikzlibrary{decorations.pathmorphing, patterns,shapes}
\usepackage{amssymb}
\usepackage{amsmath}

\begin{document}
    \noindent\makebox[\textwidth][c]{\Large A számítástudomány alapjai 2022. I. félév}
    \noindent\makebox[\textwidth][c]{\Large 6. gyakorlat}
    \begin{enumerate}
        \item Síkbarajzolhatóak-e az alábbi gráfok?
        \begin{figure}[!ht]
            \centering
            \includestandalone{../grafok/sik1}
            \includestandalone{../grafok/sik2}
            \includestandalone{../grafok/sik3}
            \includestandalone{../grafok/sik4}
            \includestandalone{../grafok/sik5}
            \includestandalone{../grafok/sik6}
        \end{figure}
        \item Egy konvex test minden lapja négyszög vagy nyolcszög és minden pontban pontosan $3$ lap találkozik. Mennyi a négyszög- és nyolcszöglapok számának különbsége?
        \item Rajzoljuk le az alábbi gráfok duálisait!
            \begin{figure}[ht]
                \centering
                \begin{subfigure}{0.2\textwidth}
                    \centering
                    \includestandalone[scale=0.85]{../grafok/dual1}
                    \caption{}
                \end{subfigure}
                \begin{subfigure}{0.2\textwidth}
                    \centering
                    \includestandalone[scale=0.85]{../grafok/dual2}
                    \caption{}
                \end{subfigure}
                \begin{subfigure}{0.2\textwidth}
                    \centering
                    \includestandalone[scale=0.85]{../grafok/dual3}
                    \caption{}
                \end{subfigure}
                \begin{subfigure}{0.2\textwidth}
                    \centering
                    \includestandalone[scale=0.85]{../grafok/dual4}
                    \caption{}
                \end{subfigure}
            \end{figure}
        \hrule
            \item Az alábbi gráfok közül melyek topologikusan izomorfak?
			\begin{figure}[!ht]
				\centering
				\begin{subfigure}{0.15\textwidth}
					\centering
					\includestandalone{../grafok/topizo1}
					\caption{}
				\end{subfigure}
				\begin{subfigure}{0.15\textwidth}
					\centering
					\includestandalone{../grafok/topizo2}
					\caption{}
				\end{subfigure}
				\begin{subfigure}{0.15\textwidth}
					\centering
					\includestandalone{../grafok/topizo3}
					\caption{}
				\end{subfigure}
				\begin{subfigure}{0.15\textwidth}
					\centering
					\includestandalone{../grafok/topizo4}
					\caption{}
				\end{subfigure}
				\begin{subfigure}{0.15\textwidth}
					\centering
					\includestandalone{../grafok/topizo5}
					\caption{}
				\end{subfigure}
				\begin{subfigure}{0.15\textwidth}
					\centering
					\includestandalone{../grafok/topizo6}
					\caption{}
				\end{subfigure}
			\end{figure}

            \item Töröljünk ki az alábbi gráfokból csúcsokat vagy éleket úgy, hogy a megmaradt gráf topologikusan izomorf legyen egy $K_5$-tel vagy egy $K_{3,3}$-mal, ahol a $K_5$ vagy $K_{3,3}$ csúcsait a fekete négyzetek jelölik!
            \begin{figure}[!ht]
                \centering
                \begin{subfigure}{0.2\textwidth}
                    \centering
                    \includestandalone{../grafok/sik_basic3}
                \end{subfigure}
                \begin{subfigure}{0.2\textwidth}
                    \centering
                    \includestandalone{../grafok/sik_basic1}
                \end{subfigure}
                \begin{subfigure}{0.2\textwidth}
                    \centering
                    \includestandalone{../grafok/sik_basic2}
                \end{subfigure}
                \begin{subfigure}{0.2\textwidth}
                    \centering
                    \includestandalone{../grafok/sik_basic4}
                \end{subfigure}
            \end{figure}
                
            \item Ujjgyakorlatok (\textit{Études}):
            \begin{itemize}
                \item Egy összefüggő, egyszerű gráf síkbarajzoltja $20$ csúcsból és $19$ élből áll. Hány tartománya van?
                \item Egy egyszerű gráfnak $8$ csúcsa van, melyek közül $3$ izolált pont. A maradék $5$ csúcs egy komponenst alkot és $4$ tartományra osztja a síkot, hány éle lehet?
                \item Egy $8$ csúcsú összefüggő egyszerű gráf $5$ tartományra osztja a síkot. Hány éle van?
                \item Egy $6$ csúcsú összefüggő egyszerű gráfnak $10$ éle van. Hány tartományra osztja a síkot a gráf síkbarajzoltja?
                \item Egy $6$ csúcsú összefüggő egyszerű gráfnak $13$ éle van. Hány tartományra osztja a síkot a gráf síkbarajzoltja?
                \item Egy egyszerű gráfnak $8$ csúcsa van, melyek közül $3$ izolált pont. A maradék $5$ csúcs egy komponenst alkot és $10$ éle van. Hány tartományra oszthatja a síkot a gráfnak egy lerajzolása?
            \end{itemize}

            \item Tegyük fel, hogy $G$ összefüggő, síkbarajzolható, és $G$ minden lapja háromszög, illetve, hogy $G^*$ minden lapja négyszög. Hány pontja és hány éle van $G$-nek?
            \item Igazoljuk, hogy ha $G$ $n$ pontú síkbarajzolható gráf, és $G$ izomorf $G^*$-gal, akkor $G$-nek $2n-2$ éle van! Tetszőleges $n>3$-ra mutassunk példát ilyen $G$-re!

            \hrule
            \item \textbf{[ZH]} Síkbarajzolhatóak-e az alábbi gráfok?
            \begin{figure}[!ht]
                \centering
                \begin{subfigure}{0.2\textwidth}
                    \centering				
                    \includestandalone[height = 3cm]{../grafok/kromatikus2}
                    \caption{ZH - 2010}
                \end{subfigure}
                \begin{subfigure}{0.2\textwidth}
                    \centering					
                    \includestandalone[height = 3cm]{../grafok/sik_2015zh}
                    \caption{ZH - 2015}
                \end{subfigure}
                \begin{subfigure}{0.25\textwidth}
                    \centering					
                    \includestandalone[height = 3cm]{../grafok/sik_2011pzh}
                    \caption{PZH - 2011}
                \end{subfigure}
                \begin{subfigure}{0.2\textwidth}
                    \centering					
                    \includestandalone[height = 3cm]{../grafok/sik_2012zh}
                    \caption{ZH - 2012}
                \end{subfigure}
            \end{figure}
            \item Igazoljuk, hogy ha a egy egyszerű $G$ gráfnak legalább $11$ csúcsa van, akkor $G$ és $\bar{G}$ közül legalább az egyik nem síkbarajzolható.
            \item Hány csúcsa van egy olyan összefüggő síkbarajzolható gráfnak, aminek három háromszöglapja, három négyszöglapja és egy ötszöglapja van?
    \end{enumerate}
\end{document}