\documentclass[a4paper, 12pt]{article}
    
    \usepackage[top=1.5cm,bottom=1.5cm,left=1cm,right=1cm]{geometry}
    
    \usepackage[utf8]{inputenc}
    \usepackage[magyar]{babel}
    
    \usepackage{standalone}
    \usepackage{tikz}
    \usetikzlibrary{positioning, graphs}
    \usetikzlibrary{graphs.standard}
    \usetikzlibrary{arrows.meta}
    \usepackage{alphalph}
    
    \usepackage{caption}
    \usepackage{subcaption}
    \usepackage{amsmath}

    \usepackage{wrapfig}
    
    \begin{document}
        \noindent\makebox[\textwidth][c]{\Large A számítástudomány alapjai 2021. I. félév}
        \noindent\makebox[\textwidth][c]{\Large 7. gyakorlat}
        
        % \noindent\centering\parbox[c][4em][c]{0.8\textwidth}{\centering Gráfok színezése, klikkméret és kromatikus szám viszonya. Páros gráf fogalma. Hálózati folyamok, Ford-Fulkerson tétel, algoritmus maximális folyam keresése, egészértékűségi lemma.}
        
        \begin{enumerate}
            % \item Állapítsuk meg az alábbi gráfok kromatikus számait! Mely gráfok párosak az alábbiak közül? Állapítsuk meg a $\nu$ (max. független élek), $\rho$ (min. lefogó élek), $\alpha$ (max. független pontok) és $\tau$ (min. lefogó pontok) paramétereket az alábbi gráfokban!
            % \begin{figure}[!h]
            %     \centering
            %     \begin{subfigure}{0.16\textwidth}
            %         \centering		
            %         \includestandalone[width=0.7\textwidth]{../grafok/k6}
            %     \end{subfigure}
            %     \begin{subfigure}{0.16\textwidth}
            %         \centering		
            %         \includestandalone[width=0.7\textwidth]{../grafok/minikromatikus1}
            %     \end{subfigure}
            %     \begin{subfigure}{0.16\textwidth}
            %         \centering
            %         \includestandalone[width=0.7\textwidth]{../grafok/minikromatikus5}
            %     \end{subfigure}
            %     \begin{subfigure}{0.16\textwidth}
            %         \centering
            %         \includestandalone[width=0.7\textwidth]{../grafok/minikromatikus2}
            %     \end{subfigure}
            %     \begin{subfigure}{0.16\textwidth}
            %         \centering
            %         \includestandalone[width=0.7\textwidth]{../grafok/minikromatikus3}
            %     \end{subfigure}
            %     \begin{subfigure}{0.16\textwidth}
            %         \centering
            %         \includestandalone[width=0.7\textwidth]{../grafok/minikromatikus4}
            %     \end{subfigure}
            % \end{figure}
            \item Adjunk meg egy-egy maximális nagyságú folyamot az alábbi hálózatokban, és bizonyítsuk be, hogy nagyobb folyam nem lehetséges!
            \begin{figure}[!h]
                \centering
                \begin{subfigure}{0.4\textwidth}
                    \centering		
                    \includestandalone[scale=0.75]{../grafok/folyam3}
                \end{subfigure}
                \begin{subfigure}{0.4\textwidth}
                    \centering
                    \includestandalone[scale=0.75]{../grafok/folyam4}
                \end{subfigure}
            \end{figure}
            \item Az alábi hálózatokban valaki már létrehozott valamilyen nagyságú folyamokat. Az $s$ csúcsok a forrásokat, a $t$-k pedig a nyelőket jelölik. Az éleken zárójelben a kapacitás látható, a zárójelen kívül pedig a folyam értéke. Rajzoljuk fel a javítógráfokat és azok mentén próbáljunk meg javítani!
            \begin{figure}[!h]
                \centering
                \begin{subfigure}{0.24\textwidth}
                    \centering		
                    \includestandalone[width=\textwidth]{../grafok/minifolyam1}
                \end{subfigure}
                \begin{subfigure}{0.24\textwidth}
                    \centering
                    \includestandalone[width=\textwidth]{../grafok/minifolyam2}
                \end{subfigure}
                \begin{subfigure}{0.24\textwidth}
                    \centering
                    \includestandalone[width=\textwidth]{../grafok/minifolyam3}
                \end{subfigure}
                \begin{subfigure}{0.24\textwidth}
                    \centering
                    \includestandalone[width=\textwidth]{../grafok/minifolyam4}
                \end{subfigure}
            \end{figure}

            \hrule
            

            % \item Állapítsuk meg az alábbi gráfokban az $\omega$, $\chi$, $\nu$, $\rho$, $\alpha$ és $\tau$ paramétereket.
            % \begin{figure}[!h]
            %     \centering
            %     \begin{subfigure}{0.24\textwidth}
            %         \centering		
            %         \includestandalone[scale=.7]{../grafok/petersen}
            %     \end{subfigure}
            %     \begin{subfigure}{0.24\textwidth}
            %         \centering
            %         \includestandalone[scale=.7]{../grafok/kromatikus2}
            %     \end{subfigure}
            %     \begin{subfigure}{0.24\textwidth}
            %         \centering
            %         \includestandalone[scale=.7]{../grafok/kromatikus1}
            %     \end{subfigure}
            %     \begin{subfigure}{0.24\textwidth}
            %         \centering
            %         \includestandalone[scale=.7]{../grafok/kromatikus3}
            %     \end{subfigure}
            % \end{figure}
            
            % \hrule

            % \item Legyen $V(G)=\{1,2,3, \ldots , 100\}$, és legyen $ij \in E(G)$, ha $|i-j|\leq 7$. Mennyi az így meghatározott $G$ gráf $\chi(G)$ kromatikus száma?
            
            % \item \textbf{[ZH-2010]} Legyenek a $G$ irányítatlan gráf csúcsai az $1, 2, \ldots, 100$ számok, az $i$ és $j$ csúcs között pedig akkor fusson él, ha $j<i$ esetén az $i-j$ szám $4$-gyel osztva $1$-et ad maradékul. Páros-e a $G$ gráf?
            
            % \item Van-e olyan $G$ gráf, amiben nincs $4$ csúcsú teljes részgráf, de $G$ mégsem színezhető ki $3$ színnel?
            
            % \item Legyenek a $G$ gráf csúcsai a sakktábla mezői. Két mező közt akkor fusson él, ha a huszár (bástya, futó, király) egy lépésben az egyik mezőről a másikra léphet. Mennyi a $G$ gráf kromatikus száma?
            
            % \item Igazoljuk, hogy ha $G$ egyszerű gráf, akkor $|E(G)|\geq \binom{\chi(G)}{2}$.
            
            % \item Legyenek $K$ és $H$ a $G$ gráf két komponense. Legyen $G'$ az a gráf, amit $G$-ből úgy kapunk, hogy $K$ minden pontját összekötjük $H$ minden pontjával. Bizonyítsuk be, hogy $\chi(G)=\max\{\chi(K),\chi(H)\}$ ill. $\chi(G')=\chi(H)+\chi(K)$.
            
            % \item \textbf{[PPZH-2010]} Tegyük fel, hogy $G$ olyan $2n$ csúcsú gráf, aminek van teljes párosítása. Határozzuk meg a komplementergráf kromatikus számát, $\chi(\bar{G})$-t.
            
            % \item \textbf{[ZH-2014]} Tegyük fel, hogy a $88$ pontú $G$ páros gráf egy lefogó élhalmaza független élekből áll. Határozzuk meg $\tau(G)$ értékét, azaz a $G$-t lefogó pontok minimális számát.
            
            % \item \textbf{[ZH-2016]} Tegyük fel, hogy valamely $G$ véges, egyszerű gráfban a lefogó ponthalmaz minimális méretére és a maximális klikkméretre $\tau(G)=\omega(G)-1$ teljesül. Igazoljuk, hogy $G$ kromatikus száma $\chi(G) = \omega(G)$.
            
            % \item Legfeljebb hány éle lehet annak az $n$ csúcsú $G$ gráfnak, amire $\chi(G)\leq 2$? És akkor, ha $\chi(G)\leq 3$?
            
            % \item \textbf{[PPZH-2012]} Legyenek a $G_n$ egyszerű gráf csúcsai az $(i,j)$ számpárok, ahol $i$ és $j$ $1$ és $n$ közötti egészek. A $G$ gráf $(i,j)$ és $(k,l)$ egymástól különböző csúcsai pontosan akkor szomszédosak, ha $i=k$ vagy $j=l$. Rajzoljuk le a $G_3$ egy áttekinthető diagramját, valamint határozzuk meg a $G_3$ kromatikus számát, $\chi(G)$-t.
            \item \textbf{[PZH-2011]} A $G=(V,E)$ irányított gráf csúcshalmaza $V=\{v_{12},v_{13},v_{14},v_{15},v_{16}\}$ és $i < j$ esetén a $v_i v_j$ él kapacitása $c(v_i v_j) = (i,j)$, ahol $(i,j)$ jelöli az $i$ és $j$ számok legnagyobb közös osztóját. Más éle a $G$-nek nincs. Ha a $v_{15}v_{16}$ él kapacitását tetszés szerint megváltoztathatjuk, mennyi lehet a $v_{12}$-ből $v_{16}$-ba vezető maximális folyam nagysága? Mekkora az a legkisebb kapacitás a $v_{15}v_{16}$ élen, amire ez a maximális folyamnagyság elérhető?
            \item Adott a $D$ irányított gráf, valamint élein egy $c$ kapacitásfüggvény. Bizonyítsuk be, hogy ha $s$, $t$, és $w$ a $D$ olyan csúcsai, hogy létezik $D$-ben $m$ nagyságú $st$-folyam és $m$ nagyságú $tw$-folyam is, akkor $D$-ben létezik $m$ nagyságú $sw$-folyam.
            \item Egy $(G, s, t, c)$ hálózatban minden él piros, fehér, vagy zöld. Ha csak a piros és fehér, vagy csak a piros és zöld, vagy csak a fehér és zöld éleket tekintjük, akkor a kapott hálózatokban a maximális nagyságú $st$-folyam nagysága $10$. Bizonyítsuk be, hogy a teljes hálózatban a maximális nagyságú $st$-folyam nagysága legalább $15$.
            % \item \textbf{[ZH-2013]} Egy $n$ csúcsú konvex sokszöget egymást nem metsző átlókkal háromszögekre bontunk (háromszögeljük). (Ezt mindig többféleképpen is meg lehet tenni.) Bizonyítsuk be, hogy akárhogyan is háromszögelünk, a kapott háromszögelésre gráfként tekintve (csúcsok a sokszög csúcsai, élek a sokszög oldalai és átlói), a kapott gráf kromatikus száma mindig ugyanannyi lesz!
            \item \textbf{[PZH-2010]} A mellékelt ábrán látható hálózatban a $12$ kapacitású $df$ él elromlott, kapacitása $0$ lett. Határozzuk meg a kapott hálózatban a maximális $st$ folyam nagyságát.
            \begin{figure}[!ht]
                \centering
                \includestandalone{../grafok/folyampzh2010}
            \end{figure}
            
            Kiderült közben, hogy a kiesett élt egy $p$ kapacitású éllel tudjuk pótolni. Határozzuk meg, hogyan függ a maximális nagyságú  $st$ folyam nagysága a $p$ paraméter értékétől!

            \item %\textbf{[Codeforces \#304]}
            Amíg a Mortal Kombat harci torna szünetel, Skorpió és néhány barátja úgy döntenek, hogy elmennek árufeltöltőnek részmunkaidőben. Első nap rögtön egy nagyon fontos feladatot kapnak. Az egyetem kólaautomatái között nem lettek pontosan szétosztva reggel a kólás üvegek, valahova több érkezett, valahol pedig hiány keletkezett. Az automaták közötti utakat az alábbi bal oldali gráf szemlélteti, két automata akkor van összekötve, ha van köztük (kétirányú) út, az egyes automaták felett számok ábrázolják a kólásüvegeket (zárójelben az, hogy pontosan mennyi kólára van szükségük). Minden egyes automatánál több harcos is állomásozik, akik az automatánál lévő üvegeket át tudják vinni egy szomszédos automatához. Szét lehet-e osztani a kólákat úgy, hogy minden automatánál pontosan annyi legyen, amennyire szükség van ott úgy, hogy a harcosoknak legfeljebb csak egyszer kelljen szállítaniuk? (A harcosok egyszerre indulnak el az automatáktól, azaz senki nem várhatja meg, hogy a másik áthozzon neki valamennyi kólát.)
            \begin{figure}[!h]
                \centering
                % \begin{subfigure}{0.4\textwidth}
                %     \centering		
                    \includestandalone[scale=0.75]{../grafok/folyam_trukk}
                % \end{subfigure}
                % \begin{subfigure}{0.4\textwidth}
                %     \centering
                %     \includestandalone[scale=0.6]{../grafok/diablo4altar}
                % \end{subfigure}
            \end{figure}

            \item Évekkel az utolsó mobiltelefonok megsemmisítése után Diablo, a Terror Ura, megszűnt halhatatlannak lenni és így a nephalemnek sikerült is legyőznie őt. A nekromaták azonban újabb tervvel rukkoltak elő, mégpedig Lilith-et, a Gyűlölet Lányát szeretnék megidézni. Ehhez a lenti gráf szerint felvázolt oltár három pontjában ($\Lambda$, $\Gamma$, $\Xi$) kell embereket feláldozni és elfolyatni a vérüket a középső $\Psi$ pontba. Legfeljebb mennyi vért tudnak befolyatni az oltár közepébe, ha az csak a megjelölt utakon haladhat, és minden út maximális véráteresztőképességét az utakon jelölt szám mutatja?
            \begin{figure}[!ht]
                \centering
                \includestandalone[scale=0.6]{../grafok/diablo4altar}
            \end{figure}
        \end{enumerate}
    \end{document}