\documentclass[a4paper,12pt]{article}
    
    \usepackage[top=1.5cm, bottom=1.5cm,left=1.5cm,right=1.5cm]{geometry}
    
    \usepackage{t1enc}
    \usepackage[utf8]{inputenc}
    \usepackage[magyar]{babel}
    \usepackage{caption}
    \usepackage{subcaption}
    
    \usepackage{standalone}
    \usepackage{tikz}
    \usetikzlibrary{positioning, graphs}
    \usetikzlibrary{graphs.standard}
    \usetikzlibrary{arrows.meta}

    \usepackage{multicol}
\begin{document}
    \noindent\makebox[\textwidth][c]{\Large Gráf alapok}
    \begin{enumerate}

        \item Az alábbi gráfok közül melyek izomorfak?
        \begin{figure}[!h]
            \centering
            \begin{subfigure}{0.15\textwidth}
                \centering
                \includestandalone{../grafok/dijkstra_small_1}
            \end{subfigure}
            \begin{subfigure}{0.15\textwidth}
                \centering
                \includestandalone{../grafok/dijkstra_small_2}
            \end{subfigure}
            \begin{subfigure}{0.15\textwidth}
                \centering
                \includestandalone{../grafok/dijkstra_small_3}
            \end{subfigure}
            \begin{subfigure}{0.15\textwidth}
                \centering
                \includestandalone{../grafok/dijkstra_small_4}
            \end{subfigure}
            \begin{subfigure}{0.15\textwidth}
                \centering
                \includestandalone{../grafok/dijkstra_small_5}
            \end{subfigure}
            \begin{subfigure}{0.15\textwidth}
                \centering
                \includestandalone{../grafok/dijkstra_small_6}
            \end{subfigure}
        \end{figure}

        \item Az alábbi gráf az olasz maffia ismerettségi körét ábrázolja. A maffiózók nevei felelnek meg a gráf csúcsainak, a köztük futó élek pedig kölcsönös ismerettséget jelentenek. Jelölje a maffiózó ismerettségét a gráfban lévő fokszáma (pl. Giorno-nak 2). Ekkor ki a legismertebb maffiózó? Jelölje a maffiózó befolyását az ismerőseinek az ismerettségének (fokszámainak) az összege (például Giorno-nak 5). Ekkor ki a legbefolyásosabb maffiózó?
        \begin{figure}[!h]
            \centering
            \includestandalone[scale = 1]{../grafok/medici}
        \end{figure}


        \item \textbf{[ZH-2014]} Legyenek a $G$ egyszerű gráf csúcsai az $1,2,\ldots,10$ számok, és két különböző csúcs között akkor fusson él, ha a két szám különbsége páratlan. Hány $4$ hosszú köre van a $G$ gráfnak?
        
        \item Bizonyítsuk be, hogy bármely $13$ ember között van olyan, aki legalább $6$ másikat ismer vagy van köztük $3$ olyan, akik páronként nem ismerik egymást. (Az ismeretség kölcsönös.)


        \item Van-e olyan egyszerű gráf, aminek a fokszámai
        \begin{enumerate}
            \item $1,2,2,3,3,3$;
            \item $1,1,2,2,3,4,4$?
        \end{enumerate}

        \item Találjuk meg (izomorfia erejéig) mindazon egyszerű gráfokat, amelyekre
        \begin{multicols}{3}
        \begin{enumerate}
            \item $n=5$, $e=2$;
            \item $n=5$, $e=3$;
            \item $n=5$, $e=7$;
            \item $n=4$, $e=5$;
            \item $n=5$, $e=8$.
        \end{enumerate}
        \end{multicols}


        \hrule

        

        \item Határozzuk meg az összes olyan véges, egyszerű $G$ gráfot, aminek nincs két azonos fokú csúcsa.
        
        \item Mutassuk meg, hogy ha $G$ véges gráf, akkor páratlan fokú pontjainak száma páros. Ha $G$ nem véges, akkor ez nem igaz.
        
        \item Mutassuk meg, hogy ha egy $G$ gráfnak $11$ csúcsa és $45$ éle van, akkor $G$-nek van olyan csúcsa, ami legalább $9$-edfokú.
        
        \item \textbf{[ZH-2015]} Tegyük fel, hogy a $G$ egyszerű gráfnak $100$ csúcsa van, melyek bármelyikének a fokszáma legalább $33$, továbbá $G$-nek van olyan csúcsa, melyből legalább $66$ él indul. Bizonyítsuk be, hogy $G$ összefüggő.
        
        \item \textbf{[ZH-2016]} A $G$ gráfnak $n+3$ csúcsa van: ebből $3$ piros $(a, b, c)$ és $n$ zöld $(v_1, v_2, \ldots, v_n)$. Két csúcs pontosan akkor szomszédos $G$-ben, ha a színük különbözik. Hány $6$ pontú kör van a $G$ gráfban?
        
        \item \textbf{[ZH-2012]} Tegyük fel, hogy a háromszöget nem tartalmazó, irányítatlan, $100$ csúcsú $G$ egyszerű gráf $4$-reguláris, azaz minden fokszáma $4$. Hány $3$-élű útja van $G$-nek?
        
        \item Legyenek $e,f$ és $g$ a $G$ egyszerű, összefüggő gráf különböző élei. Tegyük fel, hogy a $G$ gráf összefüggő marad, bármely élét is hagyjuk el, ám a $G-e-f$ és a $G-e-g$ gráfok egyike sem összefüggő. Igazoljuk, hogy ekkor a $G-f-g$ gráf sem összefüggő.
        
        \item \textbf{[ZH-2000]} Az előre megszámozott (címkézett) $n$ darab pont közé hányféleképp húzhatunk be éleket úgy, hogy egyszerű gráfhoz jussunk?
        
        \item \textbf{[ZH-2000]} Rajzoljuk le azt a gráfot, melynek pontjai a $4$ hosszú nullákból és egyesekből álló sorozatok és két csúcs akkor van éllel összekötve, ha egyik a másikból egy ,,forgatással'' megkapható, azaz ha az egyik a $(b_1,b_2,b_3,b_4)$ akkor a másik a $(b_2,b_3,b_4,b_1)$ sorozathoz tartozó pont.

        \item \textbf{[PZH-2014]} Igazoljuk, hogy ha egy $6$ csúcsú $G$ gráf fokszáma $2$, $2$, $2$, $4$, $5$, $5$, akkor $G$ nem egyszerű.

        \item \textbf{[PZH-2015]} Igazoljuk, hogy ha $v$ egy véges $G$ gráf páratlan fokú csúcsa, akkor $G$-ben van olyan út, amely $v$-t a $G$ egy másik páratlan fokú csúcsával köti össze.

        \item Bizonyítsuk be, hogy ha $G$ egyszerű gráf, akkor élei irányíthatók úgy, hogy ne jöjjön létre irányított kör.
        
        \item A $V=\{1,2, \ldots, 2n \}$ (számozott) pontokon hány olyan egyszerű $G$ gráf adható meg, melynek $2n-2$ éle van és két egyforma méretű összefüggő komponensből áll?
        
        \item Egy $n\times n$ méretű $T$ táblázatnak nincs két egyforma sora. Bizonyítsuk be, hogy $T$-nek van olyan oszlopa, amit törölve a maradék táblázatban sem lesz két egyforma sor.

        
    \end{enumerate}
\end{document}