\documentclass[a4paper, 12pt]{article}

\usepackage[utf8]{inputenc}
\usepackage[magyar]{babel}
\usepackage[top = 2.5 cm, left = 2.5 cm, right = 2.5 cm, bottom = 2.5 cm]{geometry}
\usepackage{standalone}
\usepackage{tikz}
\usetikzlibrary{positioning, graphs, matrix}
\usetikzlibrary{graphs.standard}
\usepackage{amssymb,graphicx,color}
\usepackage{caption}
\usepackage{subcaption}

\newcommand{\defi}{{\bf Def:} }
\newcommand{\megf}{{\bf Megfigyelés:} }
\newcommand{\tetel}{{\bf Tétel:} }
\newcommand{\kov}{{\bf Köv.:} }
\newcommand{\ul}{\underline}
\newcommand{\all}{{\bf Állítás:} }
\newcommand{\Z}{{\mathbb Z}}
\newcommand{\N}{{\mathbb N}}



\long\def\mo#1{}

\begin{document}
\begin{center}
{\huge A számítástudomány alapjai 2015.\ I. félév}\\ 
{\large 9. gyakorlat.
%2015.\ október 28.
\"Ossze\'all\'\i totta: Fleiner Tam\'as ({\tt fleiner@cs.bme.hu}) nyomán Nguyen Tuan Hai}
\end{center}

\noindent
{\bf\large\underline{Tudnivalók}}



\defi A $G$ gráf \emph{síkbarajzolható}, ha létezik $G$-nek olyan
diagramja, amiben az éleknek megfelelő görbék (töröttvonalak) csak
végpontokban metszhetik egymást. A síkbarajzolás a síkot
\emph{tartományokra (lapokra)} osztja. Lesz egy végtelen tartomány, az ún.\
\emph{külső} tartomány. Gömbre rajzoláson lényegében ugyanezt értjük, csak
sík helyett a gömb felszínén dolgozunk, külső tartomány nincs.

\tetel A $G$ gráf pontosan akkor síkbarajzolható, ha gömbre rajzolható.

\kov Tetszőleges konvex poliéder élhálója síkbarajzolható.

{\bf Hasznos összefüggés} Ha egy $G$ síkbarajzolt gráfnak $e$ éle van, és
az egyes tartományait $l_1,l_2,\ldots$ él határolja, akkor $2e=\sum_i
l_i$. (Ha egy $uv$ él mindkét oldalán ugyanaz a $t_i$ tartomány fekszik, akkor
$uv$-t kétszer számoljuk $l_i$-be.)

\tetel Ha $G$ sr, $n$ csúcsa, $e$ éle, $k$ komponense és $t$ tartománya
van, akkor $n+t=e+k+1$ .

\kov Ha $G$ sr, akkor bármely síkbarajzolásának ugyanannyi
tartománya van.

{(\bf Euler-formula)} Ha egy öf sr gráfnak $n$ pontja, $e$ éle és
$t$ tartománya van, akkor $n+t=e+2$.


\kov Ha $G$ egyszerű, legalább $3$-pontú, sr gráf, akkor $e\le
3n-6$ . Ha $G$-nek háromszöglapja sincs, akkor még $e\le 2n-4$ is igaz.

\kov Ha $G$ sr és egyszerű, akkor van legfeljebb $5$-ödfokú
csúcsa, azaz $\delta (G)\le 5$.

\kov Sem $K_5$, sem $K_{3,3}$ nem síkbarajzolható.

\defi A $H$ gráf a $G$ gráf \emph{soros bővítése}, ha $H$
megkapható $G$-ből úgy, hogy $G$ bizonyos éleit olyan utakkal
helyettesítjük, amelyeknek belső csúcsai különböznek egymástól és $G$ csúcsaitól.

{\bf Kuratowski tétel:} A $G$ gráf pontosan akkor sr, ha részgráfként nem
tartalmazza sem $K_{3,3}$, sem $K_5$ soros bővítését.


\defi Legyen $G=(V,E)$ síkbarajzolt gráf, legyen $V^*$ a $G$ lapjainak
halmaza.  $G^*=(V^*,E^*)$ a $G$ \emph{duálisa}, ahol $E^*=\{e^*:e\in E\}$
és $e^*$ az $e$-t határoló tartomány(oka)t összekötő él.

\defi A $Q\subseteq E(G)$ élhalmaz \emph{vágás}, ha $Q$ egy olyan élhalmaz,
hogy egyrészt $Q$ elhagyásakor $G$ szétesik (azaz komponenseinek száma
megnő), másrészt $Q$ egy legszűkebb élhalmaz ezzel a tulajdonsággal, azaz
$Q$ semelyik valódi részhalmazának elhagyásától sem esik $G$ szét. Az 
$e$ él \emph{elvágó él}, ha $\{e\}$ vágás. A $G$ gráf $e$ és $e'$ élei
\emph{soros élek}, ha $\{e,e'\}$ vágás.

\tetel Legyen  $G=(V,E)$ sr.
(1) Ha $G^*$ a $G$ duálisa, akkor $G^*$ sr és öf.

(2) $f(e):=e^*$ egy $f:E(G)\to E(G^*)$ természetes bijekciót definiál.

(3) $G$ lapjai bijektíven $G^*$ pontjainak felelnek meg.

(4) $C\subseteq E(G)$ a $G$ köre (vágása) $\iff f(C)$ $G^*$ vágása (köre).

(5) $e\in E(G)$ a $G$ hurokéle (elvágó éle) $\iff f(e)$ a $G^*$ elvágó éle
(hurokéle).

(6) $e,e'\in E(G)$ párhuzamos (soros) élek $\iff f(e), f(e')$ soros
(párhuzamos) élek.


(7) Ha $G$ öf, akkor $G=(G^*)^*$, és ekkor $G$ pontjai bijektíven $G^*$
lapjainak felelnek meg.





{\bf Ötszíntétel:} Ha $G$ síkbarajzolható, akkor $\chi(G)\le 5$.

{\bf Négyszíntétel:} Ugyanez, $4$-gyel.






\defi Ha $a,b,k\in \Z$ és $b=k\cdot a$, akkor $a$ \emph{osztója} $b$-nek
($b$ \emph{többszöröse} $a$-nak), jelölése $a\mid b$.

\defi A $p\in\Z$, $|p|>1$ szám \emph{felbonthatatlan}, ha csak $1\cdot p,
p\cdot 1, (-1)\cdot (-p)$ és $(-p)\cdot (-1)$ alakban áll elő egészek
szorzataként. (Azaz, ha $a\mid p$ és $1<|a|$, akkor $|a|=|p|$.)  A $z\in\Z$
\emph{összetett}, ha $|z|>1$ és $z$ nem felbonthatatlan. A $p\in\Z$,
$|p|>1$ szám \emph{prím}, ha $p\mid ab,~~ a,b\in\Z\Rightarrow p\mid a$ vagy
$p\mid b$.  (Egészek szorzatát csak úgy oszthatja, ha valamelyik tényezőt
osztja.)

\all Tetszőleges $1$-nél nagyobb egész szám előáll felbonthatatlan számok
szorzataként. 

{\bf A számelmélet alaptétele:} Tetszőleges $n$ egész (melyre $2\le |n|$) a
tényezők sorrendjétől és esetleges $(-1)$ tényezőktől eltekintve
egyértelműen áll elő felbonthatatlan számok szorzataként.

\kov Egy $p$ egész szám pontosan akkor felbonthatatlan, ha prím.

\defi Az $n$ \emph{kanonikus alakja} $n=\prod_{i=1}^kp_i^{\alpha_i}$, ahol a
$p_i$-k prímek, és $1\le \alpha _i\in \N$  $\forall i$.

\all %Legyen $n$ kanonikus alakja $n=\prod_{i=1}^kp_i^{\alpha_i}$. 
Egy $d>0$ egész
pontosan akkor osztója $n$-nek, ha $d$ kan.\ alakjában csak $n$
prímosztói szerepelnek, legf az $n$ kan. alakjában szereplő kitevőn.
($n=\prod_{i=1}^kp_i^{\alpha_i}\Rightarrow d=\prod_{i=1}^kp_i^{\beta_i}$,
$0\le \beta_i\le \alpha _i$.) 

\kov Ha $1<n$ kan.\ alakja $n=\prod_{i=1}^kp_i^{\alpha_i}$, akkor
$n$ poz.\ osztóinak száma $d(n)=\prod_{i=1}^k (\alpha_i+1)$.
\iffalse és $\sigma(n)=\prod_{i=1}^k\sum_{j=0}^{\alpha_i}
p_i^j=\prod_{i=1}^k\frac{p_i^{1+\alpha_i}-1}{p_i-1}$ az $n$ pozitív
osztóinak összege.
\fi



\defi Az $a$ és $b$ számok \emph{legnagyobb közös osztója} az $a$ és $b$
közös osztói közül a legnagyobb: $(a,b):=\max\{d:d\mid a, d\mid b\}$,
\emph{legkisebb közös többszörösük} pedig az $a$ és $b$ pozitív közös
többszörösei közül a legkisebb: $[a,b]:=\min\{0<d:a\mid d, b\mid d\}$. Az
$a$ és $b$ egészek \emph{relatív prímek}, ha $(a,b)=1$, azaz nincs közös
prímosztójuk (a kanonikus alakjaikban szereplő prímkek különbözők).

\all Ha $a=p_1^{\alpha_1}\cdot p_2^{\alpha_2}\cdot\ldots\cdot
p_k^{\alpha_k}$ és $b=p_1^{\beta_1}\cdot p_2^{\beta_2}\cdot\ldots\cdot
p_k^{\beta_k}$ ($\alpha_i=0$ és $\beta_i=0$ is lehet), akkor\\
$(a,b)= p_1^{\min(\alpha_1, \beta_1)}\cdot p_2^{\min(\alpha_2,
  \beta_2)}\cdot\ldots\cdot p_k^{\min(\alpha_k, \beta_k)}, 
[a,b]= p_1^{\max(\alpha_1, \beta_1)}\cdot p_2^{\max(\alpha_2,
  \beta_2)}\cdot\ldots\cdot p_k^{\max(\alpha_k, \beta_k)},$\\
valamint $ab=(a,b)\cdot [a,b]$. (Azaz a lnko-t $a$ és $b$ prímosztóit a
kisebb hatványon, a lkkt-t pedig ugyanezen prímosztókat a nagyobb hatványon
összeszorozva kapjuk meg.)

\kov Ha $d\mid a$ és $d\mid b$ közös osztó, akkor $d\mid (a,b)$.











{\bf\large\underline{Gyakorlatok}}

\begin{enumerate}
	\item Az alábbi gráfok közül melyek kaphatóak meg egy másik soros bővítésével. Azaz keressünk topologikusan izomorf gráfokat.
		\begin{figure}[h]
		\centering
			\begin{subfigure}[b]{0.15\textwidth}
				\centering
				\caption{}
				\includestandalone{topizo1}
			\end{subfigure}
			\begin{subfigure}[b]{0.15\textwidth}
				\centering
				\caption{}				
				\includestandalone{topizo2}
			\end{subfigure}
			\begin{subfigure}[b]{0.15\textwidth}
				\centering
				\caption{}				
				\includestandalone{topizo3}
			\end{subfigure}
		\centering
			\begin{subfigure}[b]{0.15\textwidth}
				\centering
				\caption{}	
				\includestandalone{topizo4}
			\end{subfigure}
			\begin{subfigure}[b]{0.15\textwidth}
				\centering
				\caption{}
				\includestandalone{topizo5}
			\end{subfigure}
			\begin{subfigure}[b]{0.15\textwidth}
				\centering
				\caption{}
				\includestandalone{topizo6}
			\end{subfigure}
		\end{figure}
	\item Egy mezőn $k$ ház és $k$ kút áll. Minden háztól pontosan $4$ (különböző) kúthoz vezet út (méghozzá közvetlenül, vagyis más házak vagy kutak érintése nélkül). Mutassuk meg, hogy biztosan van két olyan út, amelyek keresztezik egymást!
	\item Hány csúcsa van egy olyan öf síkbarajzolható gráfnak, aminek három háromszöglap-ja, három négyszöglapja és egy ötszöglapja van?
	\item Síkbarajzolhatók-e a $K_6$, $K_{4,2}$, $K_{4,3}$, $K_5 - e$, $K_{3,3} - e$ gráfok? Hát az alábbiak?
	\begin{figure}[h]
	\centering
		\begin{subfigure}[b]{0.3\textwidth}
			\centering
			\caption{}
			\includestandalone{sik1}
		\end{subfigure}
		\begin{subfigure}[b]{0.3\textwidth}
			\centering
			\caption{}				
			\includestandalone{sik2}
		\end{subfigure}
		\begin{subfigure}[b]{0.3\textwidth}
			\centering
			\caption{}				
			\includestandalone{sik3}
		\end{subfigure}
		\begin{subfigure}[b]{0.3\textwidth}
			\centering
			\caption{}	
			\includestandalone{sik4}
		\end{subfigure}		
		\begin{subfigure}[b]{0.3\textwidth}
			\centering
			\caption{}
			\includestandalone{petersen}
		\end{subfigure}
		
	\end{figure}
	\item Egy $20$-csúcsú poliédernek $12$ lapja van, mindegyik $k$ oldalú sokszög. Mennyi a $k$ értéke?
	\item Van-e olyan $9$-pontú $G$ gráf, hogy sem $G$ sem a $\overline{G}$ komplementere nem síkbarajzol-ható?
	\item Bizonyítsuk be, hogy nem létezik $5$ olyan ország, amik páronként szomszédosak!
	\item Mutassuk meg, hogy a $K_5$, $K_6$, $K_7$ és a $K_{3,3}$ gráfok mindegyike tóruszra (úszógumira) rajzolható. Bizonyítsuk be, hogy ha a $G$ gráf síkbarajzolható, és $G$-be behúzunk egy $e$ élt, akkor a kapott $G+e$ gráf tóruszra rajzolható.
	\item Igazoljuk, hogy ha egy egyszerű $G$ gráfnak legalább $11$ csúcsa van, akkor $G$ és $\overline G$ közül legalább az egyik nem síkbarajzolható.
	\item Mutassuk meg, hogy ha a $G$ síkbarajzolt gráf minden lapját páros számú él határolja, akkor $G$ páros gráf.
	\item Abszurdisztán adóhivatala egy papírfecnin szerzett értesülés nyomán szeretne felderíteni bizonyos ÁFA-csalásokat. A szövevényes bűnügy felgöngyölítéséhez elkészí-tettek egy $G$ gráfot, melynek pontjai a gyanús cégeknek felelnek meg és $G$ két csúcsa között akkor fut él, ha a két szóban forgó cég egyike számlát állított ki a másiknak. Az adatok gondos analízise nyomán az derült ki, hogy minden gyanús cégnek legalább hat másik gyanús céggel volt már közös számlázási ügye. A nyomozás sikerének pedig az a kulcsa, hogy ez a $G$ gráf átlátható legyen, azaz, hogy $G$-t úgy lehessen lerajzolni egy
dátummal, pecséttel és aláírással ellátott okmányra, hogy élek belső pontban ne keresztezzék egymást. (Ha ugyanis eredménytelen marad a próbálkozás, akkor sajnos képtelenség felderíteni az csalásokat.) Sikerül-e vajon nyakon csípni az elvetemült bűnözőket?\hspace*{0em}\hfill\hbox{(ZH '14)}
	\item Bizonyítsuk be, hogy ha egy egyszerű $G$ gráf síkbarajzolható, akkor a pontjainak legfeljebb a fele lehet $10$-nél nagyobb fokú.\hspace*{0em}\hfill\hbox{(pZH '14)}
	\item Rajzoljuk le a következő gráfok duálisát!
	\begin{figure}[h]
		\centering
		\begin{subfigure}[b]{0.3\textwidth}
			\centering
			\caption{}
			\includestandalone{dual1}
		\end{subfigure}
		\begin{subfigure}[b]{0.3\textwidth}
			\centering
			\caption{}
			\includestandalone{dual2}
		\end{subfigure}
		\begin{subfigure}[b]{0.3\textwidth}
			\centering
			\caption{}
			\includestandalone{dual3}
		\end{subfigure}
	\end{figure}
	
	\item Tfh $G$ öf, sr, és $G$ minden lapja háromszög, ill., hogy $G^*$ minden lapja négyszög. Hány pontja és hány éle van $G$-nek?
	\item Igazoljuk, hogy ha $G$ $n$ pontú sr gráf, és $G$ izomorf $G^*$-gal, akkor $G$-nek $2n-2$ éle van! Tetszőleges $n>3$-ra mutassunk példát ilyen $G$-re!
	
	\item Mely $p$ prímre lesz (a) $p+10$ és $p+14$ prím? \mo{3-as oszthatóság miatt $p=\pm 3$} (b) $p^2+2$ prím? \mo{$3$-as oszthatóság, $p=\pm 3$} (c) $p^2+4$ és $p^2+6$ prím? \mo{$5$-ös oszthatóság, $p=\pm 5$}
	\item Igazoljuk, hogy bármely hat egymást követő egész szám szorzata osztható $720$-szal.
	\item Ma van Dzsenifer születésnapja. Matekórán a tanítónénije szólt, hogy a 2013-ban  ünnepelt születésnapján az életkora osztója volt az az aktuális évszámnak. Hány éves most Dzsenifer?
	\item Bizonyítsuk be, hogy bármely öt szomszédos pozítív egész szám között van olyan, amely a másik négyhez relatív prím.
	\item Melyik az a legkisebb $n$ pozitív egész szám, amire $3\nmid n$ és $n$ osztóinak száma $d(n)=12$?
	\item Hány olyan pozitív egész szám van, ami az $n=2^3\cdot 7^5\cdot 11^2$ és $m=2^5\cdot 5^3\cdot 7\cdot 13$ számok közül legalább egynek osztója?
	\item Hány pozitív osztója van $10!$-nak?
	\item Melyik az a legkisebb pozitív egész, aminek pozitív osztói száma $10$-zel osztható?
	
\end{enumerate}
\end{document}