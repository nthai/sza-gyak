\documentclass[a4paper, 12pt]{article}

\usepackage[utf8]{inputenc}
\usepackage[magyar]{babel}
\usepackage[top = 2.5 cm, left = 2 cm, right = 2 cm, bottom = 2.5 cm]{geometry}
\usepackage{standalone}
\usepackage{tikz}
\usetikzlibrary{positioning, graphs, matrix}
\usetikzlibrary{graphs.standard}
\usepackage{amssymb,graphicx,color}
\usepackage{amsmath}
\usepackage{caption}
\usepackage{subcaption}

\newcommand{\defi}{{\bf Def:} }
\newcommand{\megf}{{\bf Megfigyelés:} }
\newcommand{\tetel}{{\bf Tétel:} }
\newcommand{\kov}{{\bf Köv.:} }
\newcommand{\ul}{\underline}
\newcommand{\all}{{\bf Állítás:} }
\newcommand{\Z}{{\mathbb Z}}
\newcommand{\N}{{\mathbb N}}



\long\def\mo#1{}

\begin{document}
\begin{center}
{\huge A számítástudomány alapjai 2015.\ I. félév}\\ 
{\large 10. gyakorlat.
%2015.\ október 28.
\"Ossze\'all\'\i totta: Fleiner Tam\'as ({\tt fleiner@cs.bme.hu}) nyomán Nguyen Tuan Hai}
\end{center}

\noindent
{\bf\large\underline{Tudnivalók}}
\defi Ha $a,b,k\in \Z$ és $b=k\cdot a$, akkor $a$ \emph{osztója} $b$-nek
($b$ \emph{többszöröse} $a$-nak), jelölése $a\mid b$.

\defi A $p\in\Z$, $|p|>1$ szám \emph{felbonthatatlan}, ha csak $1\cdot p,
p\cdot 1, (-1)\cdot (-p)$ és $(-p)\cdot (-1)$ alakban áll elő egészek
szorzataként. (Azaz, ha $a\mid p$ és $1<|a|$, akkor $|a|=|p|$.)  A $z\in\Z$
\emph{összetett}, ha $|z|>1$ és $z$ nem felbonthatatlan. A $p\in\Z$,
$|p|>1$ szám \emph{prím}, ha $p\mid ab,~~ a,b\in\Z\Rightarrow p\mid a$ vagy
$p\mid b$.  (Egészek szorzatát csak úgy oszthatja, ha valamelyik tényezőt
osztja.)

\all Tetszőleges $1$-nél nagyobb egész szám előáll felbonthatatlan számok
szorzataként. 

{\bf A számelmélet alaptétele:} Tetszőleges $n$ egész (melyre $2\le |n|$) a
tényezők sorrendjétől és esetleges $(-1)$ tényezőktől eltekintve
egyértelműen áll elő felbonthatatlan számok szorzataként.

\kov Egy $p$ egész szám pontosan akkor felbonthatatlan, ha prím.

\defi Az $n$ \emph{kanonikus alakja} $n=\prod_{i=1}^kp_i^{\alpha_i}$, ahol a
$p_i$-k prímek, és $1\le \alpha _i\in \N$  $\forall i$.

\all %Legyen $n$ kanonikus alakja $n=\prod_{i=1}^kp_i^{\alpha_i}$. 
Egy $d>0$ egész
pontosan akkor osztója $n$-nek, ha $d$ kan.\ alakjában csak $n$
prímosztói szerepelnek, legf az $n$ kan. alakjában szereplő kitevőn.
($n=\prod_{i=1}^kp_i^{\alpha_i}\Rightarrow d=\prod_{i=1}^kp_i^{\beta_i}$,
$0\le \beta_i\le \alpha _i$.) 

\kov Ha $1<n$ kan.\ alakja $n=\prod_{i=1}^kp_i^{\alpha_i}$, akkor
$n$ poz.\ osztóinak száma $d(n)=\prod_{i=1}^k (\alpha_i+1)$.
\iffalse és $\sigma(n)=\prod_{i=1}^k\sum_{j=0}^{\alpha_i}
p_i^j=\prod_{i=1}^k\frac{p_i^{1+\alpha_i}-1}{p_i-1}$ az $n$ pozitív
osztóinak összege.
\fi



\defi Az $a$ és $b$ számok \emph{legnagyobb közös osztója} az $a$ és $b$
közös osztói közül a legnagyobb: $(a,b):=\max\{d:d\mid a, d\mid b\}$,
\emph{legkisebb közös többszörösük} pedig az $a$ és $b$ pozitív közös
többszörösei közül a legkisebb: $[a,b]:=\min\{0<d:a\mid d, b\mid d\}$. Az
$a$ és $b$ egészek \emph{relatív prímek}, ha $(a,b)=1$, azaz nincs közös
prímosztójuk (a kanonikus alakjaikban szereplő prímkek különbözők).

\all Ha $a=p_1^{\alpha_1}\cdot p_2^{\alpha_2}\cdot\ldots\cdot
p_k^{\alpha_k}$ és $b=p_1^{\beta_1}\cdot p_2^{\beta_2}\cdot\ldots\cdot
p_k^{\beta_k}$ ($\alpha_i=0$ és $\beta_i=0$ is lehet), akkor\\
$(a,b)= p_1^{\min(\alpha_1, \beta_1)}\cdot p_2^{\min(\alpha_2,
  \beta_2)}\cdot\ldots\cdot p_k^{\min(\alpha_k, \beta_k)}, 
[a,b]= p_1^{\max(\alpha_1, \beta_1)}\cdot p_2^{\max(\alpha_2,
  \beta_2)}\cdot\ldots\cdot p_k^{\max(\alpha_k, \beta_k)},$\\
valamint $ab=(a,b)\cdot [a,b]$. (Azaz a lnko-t $a$ és $b$ prímosztóit a
kisebb hatványon, a lkkt-t pedig ugyanezen prímosztókat a nagyobb hatványon
összeszorozva kapjuk meg.)

\kov Ha $d\mid a$ és $d\mid b$ közös osztó, akkor $d\mid (a,b)$.

\tetel Végtelen sok prímszám van. Avagy: bármely $n\in\N$-re létezik
$n$-nél nagyobb prím.

\all Minden $n$-re van legalább $n$ méretű hézag szomszédos prímek között,
azaz létezik $n$ egymást követő összetett szám.

\defi Az $1$ és $x$ közé eső prímek számát $\pi(x)$ jelöli.

{\bf Nagy prímszám tétel:} $\lim_{n\to\infty}\frac{\pi(n)}{\frac n{\ln
    n}}=1$, azaz $\pi(x)\sim \frac x{\ln x}$.

{\bf Csebisev tétel:} Minden $1\le n$-re létezik $p$ prím, melyre $n\le
p\le2n$. ($\pi(2x)>\pi(x)$ ha $x\ge 1$.)

{\bf Dirichlet tétel:} Ha $(a,d)=1$, akkor az $a,a+d,a+2d,\ldots$ számtani
sorban $\infty$ %végtelen 
sok prím van.


\all Ha $d\mid a$ akkor $(d\mid b \iff d\mid b-a)$.

\kov $a$ és $b$ közös osztói azonosak $a$ és $b-a$ közös osztóival.

\kov Ha $a,b$ egészek, akkor $(a,b)=(a,b-a)$. Sőt: tetszőleges $k$ egészre
$(a,b)=(a,b-ka)$. 

\defi $a=q\cdot b +r$ \emph{maradékos osztás}, ha $q\in \Z$ a hányados és
$0\le r< b$ a maradék.\\
Pl a $-17$ szám $5$-ös osztási maradéka pl. $3$, de ugyanennyi a $-4$-es
osztási maradéka is.

{\bf Euklideszi algoritmus} Input: $a,b \in \N$. Output: $(a,b)$.\\
Tfh $a\ge b$. Legyen $a_0:=a, a_1:=b$ és $a_0=a_1h_1+a_2$, ahol $0\le
a_2<a_1$ (maradékos osztás). Általában $a_{i-1}=a_{i}h_{i}+a_{i+1}$, ahol
$0\le a_{i+1}<a_{i}$. Ha $a_{k+1}=0$, akkor $(a_0,a_1)=(a_1,a_2)=(a_2,a_3)=
\ldots =(a_k,0)=a_k$ a keresett lnko.
\hfill
\kov Ha $a,b\in \Z_+$,
%(nem mind nullák), 
akkor létezik $k,l\in \Z$,
melyre $(a,b)=ka+lb$.


\defi Tetszőleges rögzített $m>1$ esetén a $\Z$ halmaz előáll $m$ db halmaz
diszjunkt uniójaként, az egyes halmazokban azok a számok vannak, amelyek
$m$-mel osztva ugyanannyi maradékot adnak. (Konkrétan az $i$-dik ilyen
részhalmazba a $\{i+km:k\in \Z\}$ számok tartoznak.) E részhalmazok az \emph{$m$
szerinti maradékosztályok}.

\megf $a$ és $b$ ugyanabba az $m$ szerinti
maradékosztályba tartoznak $\iff m\mid a-b$.

\all Ha $a\equiv b (m)$ ($a$ és $b$ ugyanabból a
% mod $m$ 
maradékosztályból valók) akkor $(a,m)=(b,m)$.

\kov Ha $(a,m)=1$, akkor az $a$ maradékosztályának bármely eleme relatív
prím a modulushoz.

\defi $a,b,m\in \Z$ esetén $a\equiv b(\mathrm{mod}\ m)$ 
(\emph{$a$ kongruens $b$ modulo $m$}, röviden $a\equiv b(m)$), ha $m\mid
a-b$. Két szám tehát pontosan akkor kongruens egymással modulo $m$, ha
ugyanabba az $m$ szerinti maradékosztályba tartoznak, azaz, ha $m$-mel
osztva ugyanannyi maradékot adnak. 

{\bf Kongruenciák tulajdonságai:} $\forall a,b,c,d,m\in\Z$-re
\hfil
(1) $a\equiv a(m)$,
\hfil
(2) $a\equiv b(m)\Rightarrow b\equiv a(m)$\\
(3) $a\equiv b(m), b\equiv c(m)\Rightarrow a\equiv c(m)$\hfil
(4) $a\equiv b(m)$, $c\equiv d(m)\Rightarrow a+c\equiv b+d(m)$, $ac\equiv
bd(m)$ \\
(5) $a\equiv b(m)\iff ac\equiv bc (mc)$\hfil
(6) $ad\equiv bd(m)\Rightarrow a\equiv b\left(\frac m{(m,d)}\right)$

\defi
Az $\{a_1,a_2,\ldots,a_m\}\subset \Z$ halmaz \emph{teljes
maradékrendszer modulo $m$} (röviden \emph{tmr mod $m$}), ha minden mod $m$
maradékosztályból pontosan egy elemet tartalmaz, azaz $a_i\equiv
a_j(m)\Rightarrow i=j$. (Pl.\ $\{2015,-444, 42,13, 999\}$ tmr mod $5$, vagy 
$\{1,2,\ldots,m\}$ tmr mod $m$.)



\defi Az $ax\equiv b(m)$ neve \emph{lineáris kongruencia}, ha $a,b,m$
adottak ($m\ge 2$) és $x$ pedig ismeretlen. A lineáris kongruencia
megoldásai mindazon egész számok, melyeket $x$ helyébe helyettesítve a
kongruenciát igazzá teszik.


 {\bf Tétel (lineáris kongruenciák megoldása):} Az $ax\equiv b(m)$
kongruencia megoldható $\iff$ $(a,m)\mid b$. Ekkor $(a,m)$ db
modulo $m$ maradékosztály a megoldás. 
%Ha $(a,m)=1$, akkor a fenti kongruencia megoldása $x\equiv
%a^{\varphi(m)-1}b(m)$ . 

{\bf Konkrét lineáris kongruenciák gyakorlati megoldása} Ha van megoldás,
akkor először leosztunk az $(a,m)$ lnko-val, így feltehető, hogy $(a,m)=1$.

\ul{I. módszer}:
ügyeskedés ekvivalens az átalakítások segítségével. Az $ax\equiv b(m)$ lin
kongruenciában\\
(1) $a$-t vagy a $b$-t vele kongruens, alkalmas másik számmal helyettesítjük,\\
(2) ha $d=(a,b)>1$, akkor osztunk (az $m$ modulust is $(m,d)$-vel),\\
(3) az $m$ {\bf modulushoz relatív prímmel} szorzunk (és a
modulust nem bántjuk).\\
Az átalakítások során a cél az $x$ ismeretlen $a$
együtthatója abszolút értékének csökkentése $1$-ig.

\ul{II. módszer}: az Euklideszi algoritmus mintájára. Az $ax\equiv b(m)$
kongruenciát kiegészítjük az $mx\equiv 0(m)$ kongruenciával egy
kongruenciarendszerré, és ezt oldjuk meg. Egy lépésben abból a
kongruenciából, ahol nagyobb az $x$ együtthatója kivonjuk a másikat és e
különbségre cseréljük a nagyobb együtthatós kongruenciát. Ezt a lépést
addig végezzük, amig valamelyik kongruenciában $x$ együtthatója $1$ nem
lesz. Az így kapott kongruencia adja a megoldást.\\



\newpage





{\bf\large\underline{Gyakorlatok}}

\begin{enumerate}
	
	\item Mely $p$ prímre lesz (a) $p+10$ és $p+14$ prím? \mo{3-as oszthatóság miatt $p=\pm 3$} (b) $p^2+2$ prím? \mo{$3$-as oszthatóság, $p=\pm 3$} (c) $p^2+4$ és $p^2+6$ prím? \mo{$5$-ös oszthatóság, $p=\pm 5$}
	\item Igazoljuk, hogy bármely hat egymást követő egész szám szorzata osztható $720$-szal.
	\item Ma van Dzsenifer születésnapja. Matekórán a tanítónénije szólt, hogy a 2013-ban  ünnepelt születésnapján az életkora osztója volt az az aktuális évszámnak. Hány éves most Dzsenifer?
	\item Bizonyítsuk be, hogy bármely öt szomszédos pozítív egész szám között van olyan, amely a másik négyhez relatív prím.
	\item Melyik az a legkisebb $n$ pozitív egész szám, amire $3\nmid n$ és $n$ osztóinak száma $d(n)=12$?
	\item Hány olyan pozitív egész szám van, ami az $n=2^3\cdot 7^5\cdot 11^2$ és $m=2^5\cdot 5^3\cdot 7\cdot 13$ számok közül legalább egynek osztója?
	\item Hány pozitív osztója van $10!$-nak?
	\item Melyik az a legkisebb pozitív egész, aminek pozitív osztói száma $10$-zel osztható?
	\item \textbf{[ZH-1999]} Relatív prím-e a következő két szám: $2^{100}-1$ és $3^{100}-1$?
	\item Számítsuk ki a $(372,504)$ ill.\ $(612,834)$ legnagyobb közös osztókat.
	\item Igazoljuk, hogy tetszőleges $n$ szám $9$-es osztási maradéka megegyezik a $10$-es számrendszer-ben felírt alakjában szereplő számjegyei összegének
$9$-es maradékával.
	\mo{Azt kell megmutatni, hogy tetszőleges $a_0,a_1,\ldots, a_n$ esetén $10^na_n+10^{n-1}a_{n-1}+\ldots+10a_1+a_0\equiv a_n+a_{n-1}+\ldots+a_1+a_0(9)$. Márpedig ez következik abból, hogy $10^k\equiv 1^k(9)$, tehát $10^na_n+10^{n-1}a_{n-1}+ \ldots + 10a_1+a_0\equiv 1^n\cdot a_n+1^{n-1}\cdot a_{n-1}+\ldots+1\cdot a_1+a_0(9)$.}
	\item Igazoljuk, hogy tetszőleges $10$-es számrendszerben felírt $a_na_{n-1}\ldots a_1a_0$ szám $11$-es osztási maradéka megegyezik az $a_1-a_2+a_3\ldots \pm a_n$ szám $11$-es maradékával.
	\mo{Azt kell megmutatni, hogy tetszőleges $a_0,a_1,\ldots, a_n$ esetén $10^na_n+10^{n-1}a_{n-1}+\ldots+10a_1+a_0\equiv \sum_{i=0}^n(-1)^i\cdot a_i
(11)$, és éppen ezt kellett igazolnunk.}
	\item Oldogassunk lineáris kongruenciákat. Pl:
%\begin{enumerate}
%\item
(a) $202x\equiv 157(203)$,
\hfil
%\item
(b) $309x\equiv 451(617)$
\\
%\item
(c)$5x\equiv 13 (137)$,
\hfil
%\item
(d) $113x\equiv 77 (120)$,
\hfil
%\item
(e) $11x\equiv 12(18)$, 
\\
%\item
(f) $14x-4\equiv 80(21)$
\hfil
%\item
(g) $49^{49}x\equiv3(15)$
\hfil 
%\item
(h) $3^{80}x\equiv 23(100)$
%\end{enumerate}
	\mo{Ld NESZ, de az utolsó kettőhöz kell még EF is, ami csak a jövő héten jön.}
	\item Dzsúlió már régóta gyűjt nagy álmára, hogy volt barátnője, Vanessza mobiltelefonon őrzőtt arcképét a bicepszére tetováltassa. Legjobb barátja, Rodzser tanácsára, míg össze nem jön az ehhez szükséges 35000 forint, átváltja az ezer forintosokban tartott megtakarítását euróra, amit a Rodzser által ajánlott Rikárdótól (az ismeretségre tekintettel) szuperkedvezményes 330 Ft-os árfolyamon vesz meg. Miután Rikárdó centekkel nem foglalkozik, Dzsúliónak éppen 140 Ft marad a megtakarításából, amiből Rodzserrel közösen lottószelvényt vesznek azzal, hogy a nyereményt majd felezik. Hány euró boldog birtokosának mondhatja magát Dzsúlió a sikeres tranzakció után?
	\mo{Ha $x$ a válasz, akkor $1000\mid 330x+140$, azaz $330x\equiv -140 (1000)$ adódik. $10$-zel osztás után $33x\equiv -14 (100)$, amit ha $3$-mal szorozva $99x\equiv -42 (100)$, azaz $-x\equiv -42(100)$, tehát $x\equiv 42(100)$. Mivel Dzs megtakarítása kevesebb 35000-nél, ezért $0\le x\le 350000/330<110$, tehát egyedül az $x=42$ lehet helyes válasz. Végig ekvivalens átalakításokkal dolgoztunk, ezért nem kell ellenőrizni.}
	\item \textbf{[ZH-2005]} Bizonyítsuk be, hogy ha az $n>1$ számnak $2005$ osztója van, akkor $n$ nem lehet egy egész szám $5$-dik hatványa.
	\item \textbf{[ZH-2006]} Bizonyítsd be, hogy $3^{931}+5^{930}-52$ osztható $2006$-tal! (Segítségül: $2006=2\cdot 17 \cdot 59$.)
	\item \textbf{[ZH-2006]} Hány pozitív közös osztója van a $4422$ és a $2244$ számoknak?
	\item \textbf{[ZH-2006]} Mi az utolsó jegye a $117^{177}$ számnak $17$-es számrendszerben?
	\item \textbf{[ZH-2007]} Melyik az a legkisebb pozitív egész szám, melynek tízes számrendszerbeli alakjában a számjegyek szorzata $120$?
	\item \textbf{[ZH-2008]} Igazoljuk, hogy ha $m$ és $n$ pozitív egészek, akkor $d(n)d(m)=d(\text{lnko}(n,m) \cdot d(\text{lkkt}(n,m)))$ teljesül, ahol $d(k)$ a $k$ pozitív osztóinak sázmát, $\text{lnko}(n,m)$ és $\text{lkkt}(n,m)$ pedig rendre az $n$ és $m$ legnagyobb közös osztóját ill. legkisebb közös többszörösét jelölik.
	\item Legyen $F_0=0, F_1=1$, és $n\ge 2$ esetén az $n$-dik Fibonacci szám ,$F_n=F_{n-1}+F_{n-2}$. Igazoljuk, hogy $F_n$ és $F_{n+1}$ relatív prímek.
	\mo{$(F_{n+1},F_n)=(F_{n+1}-F_n,F_n)=(F_{n-1},F_n)=(F_n,F_{n-1}=\ldots=(F_1,F_0)= (1,0)=1$}
	\item Igazoljuk, hogy az Euklideszi algoritmusban $2a_{i+2}\le a_i$. Módosítsuk úgy az algoritmust, hogy abban csak az összeadásra és a  $2$-vel osztásra legyen szükség, maradékos osztásra pedig ne.
	\mo{$a_i=a_{i+1}\cdot h_{i+1}+a_{i+2}$, ahol $0\le a_{i+2}<a_{i+1}$. Ha $a_{i+1}\le a_i/2$, akkor kész is vagyunk, különben $a_{i+2}=a_i-a_{i+1}\le a_i/2$, így is jó. Ha nem akarunk maradékosan osztani, akkor ha $a$ és $b$ párosak, akkor $(a,b)=2(a/2,b/2)$, ha $a$ ps, $b$ ptn, akkor $(a,b)=(a/2,b)$,
végül ha mindkettő ptn és $a>b$, akkor $(a,b)=((a+b)/2,b)$.}
	\item Mi a $8$-as oszthatósági szabály $9$-es számrendszerben?
	\mo{kb az, mint a $9$-es $10$-esben.}
	\item Igazoljuk a $7$-tel való oszthatóság ellenőrzésére szolgáló alábbi módszer helyességét. Az $n$ szám pontosan akkor osztható $7$-tel, ha $7$-tel
osztható az a szám, amit $n$ tízes számrend-szerbeli alakjából úgy kapunk, hogy az utolsó számjegy levágásaval kapott számból levonjuk az utolsó számjegy kétszeresét. Pl.\ $2002$ pontosan akkor osztható $7$-tel, ha $200-2\cdot 2=196$ osztható $7$-tel. Ez pedig igaz, hisz $7\mid 19-2\cdot 6=7$, tehát $7\mid 2002$.
	\mo{Ha az utolsó számjegy $b$ és a levágásával kapott szám pedig $a$, akkor $7\mid 10a +b \iff 10a+b \equiv 0 (7) \iff 3a+b\equiv 0 (7)\iff 6a+2b\equiv 0(7) \iff -a+2b\equiv 0(7)\iff 7\mid -a+2b \iff 7\mid a-2b$, nekünk pedig pontosan ezt kellett igazolnunk.}
  \item Igazoljuk a $23$-mal való oszthatóság ellenőrzésére szolgáló alábbi módszer helyességét. Az $n$ szám pontosan akkor osztható $23$-mal, ha $23$-mal osztható az a szám, amit $n$ tízes számrendszerbeli alakjából úgy kapunk, hogy az utolsó két számjegy levágásaval kapott számhoz hozzáadjuk az utolsó
két számjegy alkotta szám háromszorosát. Pl.\ $2024$ pontosan akkor osztható $23$-mal, ha $20+3\cdot 24=92$ osztható $23$-mal. Ez igaz, tehát $23\mid
2024$.
	\mo{Ha az utolsó két számjegy alkotta szám $b$ és a $b$ levágásával kapott szám pedig $a$, akkor $23\mid 100a +b \iff 100a+b \equiv 0 (23) \iff 8a+b\equiv 0 (23)\iff 24a+4b\equiv 0(23) \iff a+4b \equiv 0(23)\iff 23\mid a+4b$}
	\item Alkossunk gyors módszert az előző feladatok mintájára, amellyel egy szám $17$-es oszthatóságát tudjuk eldönteni.
	\mo{Az utolsó jegy $5$-szörösét vonjuk le az utolsó jegy törlésével keletkező számból.}

\end{enumerate}
\end{document}