\documentclass[a4paper]{article}

\usepackage{amsmath}
\usepackage[top = 2cm, bottom = 1.5cm, left = 1cm, right = 1cm]{geometry}

\usepackage[utf8]{inputenc}
\usepackage[magyar]{babel}
\usepackage{tikz}
\usepackage{standalone}

\usetikzlibrary{positioning, graphs}
\usetikzlibrary{graphs.standard}

\begin{document}
	\noindent\makebox[\textwidth][c]{\Large A számítástudomány alapjai 2015. I. félév} \\
	\noindent\makebox[\textwidth][c]{\Large 2. ZH konzultáció}
	\\
	\\
	\framebox{\parbox{\dimexpr\linewidth-2\fboxsep-2\fboxrule}{\textbf{A 2. ZH főbb témakörei:} gráfok színezése (klikkek, kromatikus szám); folyamok (max. folyam, min. vágás, javító utas algoritmus); páros gráfok (párosítások, Hall-tétel, alternáló utas algoritmmus); független/lefogó pont-/élhalmzok (Kőnig-tétel, Gallai-tétel); síkbarajzolhatóság (szükséges feltételek, Kuratowski-tétel, Euler-féle poliédertétel, topologikus izomorfia); oszthatóság (osztók száma, lnko, lkkt); kongruenciák; \textit{...és minden más, ami szerepelt előadáson.}}}
	
\begin{enumerate}
	\item \textbf{[ZH-1999]} Egy négyzetrácsban legyen egy lépés, hogy egy négyzetből átmehetünk egy vele közös éllel rendelkező négyzetbe. A $100$-szor $100$-as rácsban mi az a legkevesebb szín, amivel a négyzetek kiszínezhetők úgy, hogy az egymástól pontosan két lépéssel elérhető négyzetek színe különböző?
	\item \textbf{[ZH-199]} Bizonyítsd be, hogy ha $G$ egy $n$ csúcsú egyszerű reguláris gráf, akor $\chi(G)+\chi(\bar{G})\leq n+1$. ($\bar{G}$ a $G$ gráf komplementerét jelöli.)
	\item \textbf{[ZH-1999]} Legyen $G$ egy egyszerű, összefüggő síkbarajzolható gráf, amelynek minden tartományát pontosan $6$ él határolja. Bizonyítsd be, hogy a $G$ gráfnak van legfeljebb másodfokú csúcsa!
	\item \textbf{[ZH-2005]} Bizonyítsuk be, hogy ha az $n>1$ számnak $2005$ osztója van, akkor $n$ nem lehet egy egész szám $5$-dik hatványa.
	\item \textbf{[ZH-2005]} Oldjuk meg a $21x\equiv 33\ (69)$ kongruenciát!
	\item \textbf{[ZH-2006]} Síkbarajzolható-e az alábbi gráf?
		\begin{figure}[h]
			\centering
			\includestandalone{sik5}
		\end{figure}
	\item \textbf{[ZH-2006]} Bizonyítsd be, hogy $3^{931}+5^{930}-52$ osztható $2006$-tal! (Segítségül: $2006=2\cdot 17 \cdot 59$.)
	\item \textbf{[ZH-2006]} Hány pozitív közös osztója van a $4422$ és a $2244$ számoknak?
	\item \textbf{[PZH-2006]} Határozd meg az alábbi hálózatban a maximális folyam értékét!
		\begin{figure}[h]
			\centering
			\includestandalone[scale = 1.5]{folyampzh2006}
		\end{figure}
	\item \textbf{[PHZ-2006]} Az alábbi $G$ gráfban határozd meg az $\alpha(G)$, $\tau(G)$, $\nu(G)$ és $\rho(G)$ értékeket!
		\begin{figure}[h]
			\centering
			\includestandalone{paramgrafpzh2006}
		\end{figure}
	\item \textbf{[ZH-2007]} Melyik az a legkisebb pozitív egész szám, melynek tízes számrendszerbeli alakjában a számjegyek szorzata $120$?
	\item \textbf{[PZH-2007]} Legyen $G$ az a gráf, melynek ponthalmaza három diszjunkt, $100$ pontú ponthalmaz uniója: $V(G)=V_1 \cup V_2 \cup V_3$. Pontosan akkor kötünk össze két különböző pontot, ha vagy mindkettő $V_1$-ben van, vagy két különböző halmazban vannak. (Hurokélek és párhuzamos élek nincsenek.) Mennyi $\alpha(G)$, $\nu(G)$, $\tau(G)$ és $\rho(G)$?
	\item \textbf{[PZH-2007]} Síkbarajzolható-e az alábbi gráf?
		\begin{figure}[h]
			\centering
			\includestandalone{sik6}
		\end{figure}
	\item \textbf{[PZH-2007]} Tegyük fel, hogy $G$ olyan gráf, amelyre $\chi(G)=k\geq 2$, de minden $v$ pontjára teljesül, hogy $\chi(G-v)=k-1$. Bizonyítsuk be, hogy a minimális fokszám legalább $k-1$.
	\item \textbf{[PZH-2007]} Hány olyan osztója van a $2^{14}\cdot 3^9\cdot 5^8\cdot 7^{10} \cdot 11^3$ számnak, amelyik osztható $210$-zel?
	\item \textbf{[ZH-2008]} Igazoljuk, hogy ha $m$ és $n$ pozitív egészek, akkor $d(n)d(m)=d(\text{lnko}(n,m) \cdot d(\text{lkkt}(n,m)))$ teljesül, ahol $d(k)$ a $k$ pozitív osztóinak számát, $\text{lnko}(n,m)$ és $\text{lkkt}(n,m)$ pedig rendre az $n$ és $m$ legnagyobb közös osztóját ill. legkisebb közös többszörösét jelölik.
	\item \textbf{[PZH-2008]} A házassági tanácsadón várakozó $n$ házaspár a kirakott újságokat szeretné böngészni. Tudjuk, hogy mindenkit legalább $n$ újság érdekel a választékból, de nincs olyan újság, ami valamelyik házaspár mindkét tagját érdekelné. Igazoljuk, hogy mind a $2n$ várakozó egyszerre találhat magának olvasnivalót.
	\item \textbf{[PZH-2008]} Bizonyítsuk be, hogy tetszőleges $3$-kromatikus, $100$ csúcsú $G$ gráfnak van $67$ olyan csúcsa, amik páros gráfot feszítenek.
	\item \textbf{[ZH-2010]} Legyenek a $G$ irányítatlan gráf csúcsai az $1,2,\ldots , 100$ számok, az $i$ és $j$ csúcs között pedig akkor fusson él, ha $j<i$ esetén az $i-j$ szám $4$-gyel osztva $1$-et ad maradékul. Páros-e a $G$ gráf?
\end{enumerate}
\end{document}